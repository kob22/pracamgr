\newpage
\begin{center}
\large \bf
Meta-metody służące do poprawy jakości klasyfikacji
\end{center}

\section*{Streszczenie}
W pracy przedstawiono meta-metody bagging, boosting i stacking oraz zbadano ich wpływ na poprawę jakości klasyfikacji danych. Metody te przetestowano na prawdziwych i zróżnicowanych zbiorach danych. 
Pokazano w jaki sposób należy oceniać klasyfikator, aby uzyskać wiarygodne wyniki. W pracy zaprezentowano także algorytmy służące do równoważenia klas w zbiorach o dużej przewadze w liczebności jednej klasy.
Algorytmy te porównano oraz zbadano ich współpracę z meta-metodami. Przedstawiono także sposób użycia tych algorytmów ze sprawdzianem krzyżowym. W pracy zaproponowano klasyfikator ekspercki oraz meta-klasyfikator, które zostały napisane oraz przetestowane na prawdziwych zbiorach danych.

\bigskip
{\noindent\bf Słowa kluczowe:} klasyfikacja, meta-metody, meta-klasyfikator, niezrównoważone zbiory danych


