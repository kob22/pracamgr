%--------------------------------
% praca dyplomowa
% Konrad Ziaja
% 2017
%--------------------------------


\documentclass[a4paper,12pt,twoside,openright]{report}
%
% Wzorzec pracy dyplomowej
% J. Starzynski (jstar@iem.pw.edu.pl) na podstawie pracy dyplomowej
% mgr. inż. Błażeja Wincenciaka
% Wersja 0.1 - 8 października 2016
%

%%%
% PAKIETY
%%%

\usepackage{polski}
\usepackage{helvet}
\usepackage[T1]{fontenc}
%\usepackage{anyfontsize}
\usepackage[utf8]{inputenc}
\usepackage[pdftex]{graphicx}
\usepackage{tabularx}
\usepackage{array}
\usepackage[polish]{babel}
%\usepackage{subfigure}
\usepackage{amsfonts}
\usepackage{verbatim}
\usepackage{indentfirst}
\usepackage[pdftex]{hyperref}
\usepackage{prmag2017}
\usepackage{todonotes} 
\usepackage{multirow}
% rozmaite polecenia pomocnicze
% gdzie rysunki?
\newcommand{\ImgPath}{.}

% oznaczenie rzeczy do zrobienia/poprawienia
\newcommand{\TODO}{\textbf{TODO}}


% wyroznienie slow kluczowych
\newcommand{\tech}{\texttt}

% na oprawe (1.0cm - 0.7cm)*2 = 0.6cm
% na oprawe (1.1cm - 0.7cm)*2 = 0.8cm
%  oddsidemargin lewy margines na nieparzystych stronach
% evensidemargin lewy margines na parzystych stronach
\def\oprawa{1.05cm}
\addtolength{\oddsidemargin}{\oprawa}
\addtolength{\evensidemargin}{-\oprawa}

\setlist{listparindent=\parindent, parsep=\parskip} % potrzebuje enumitem
\newcommand{\specialcell}[2][c]{%
	\begin{tabular}[#1]{@{}c@{}}#2\end{tabular}}
    




%%%%%%%%%%%%%%% Strona Tytułowa %%%%%%%%%%%%%%%%%
% To trzeba wypelnic swoimi danymi
\title{Meta-metody służące do poprawy jakości klasyfikacji}

% autor
\author{Konrad Ziaja}
\nrindeksu{272170}

\opiekun{dr inż. Łukasz Skonieczny}
\rok{2017}
\terminwykonania{XX lutego 2017} % data na oświadczeniu o samodzielności




% To sa domyslne wartosci
% - mozna je zmienic, jesli praca jest pisana gdzie indziej niz w~ZETiIS
% - mozna je wyrzucic jesli praca jest pisana w~ZETiIS
\miasto{Warszawa}
\uczelnia{POLITECHNIKA WARSZAWSKA}
\wydzial{Wydział Elektroniki i~Technik Informacyjnych}
\instytut{Instytut Informatyki}
\zaklad{Zakład Systemów Informacyjnych}
\kierunekstudiow{INFORMATYKA}
\specjalnosc{Inżynieria Systemów Informatycznych}

\pracamagisterska
%%% koniec od P.W

\opinie{%
	\input{opiniaopiekuna.tex}
	\newpage
	\input{recenzja.tex}
}

\streszczenia{
\cleardoublepage
	\input{karta_informacyjna.tex}
\cleardoublepage
	\newpage
\begin{center}
\large \bf
Meta-metody służące do poprawy jakości klasyfikacji
\end{center}

\section*{Streszczenie}
W pracy przedstawiono meta-metody bagging, boosting i stacking oraz zbadano ich wpływ na poprawę jakości klasyfikacji danych. Metody te przetestowano na prawdziwych i zróżnicowanych zbiorach danych. 
Pokazano w jaki sposób należy oceniać klasyfikator, aby uzyskać wiarygodne wyniki. W pracy zaprezentowano także algorytmy służące do równoważenia klas w zbiorach o dużej przewadze w liczebności jednej klasy.
Algorytmy te porównano oraz zbadano ich współpracę z meta-metodami. Przedstawiono także sposób użycia tych algorytmów ze sprawdzianem krzyżowym. W pracy zaproponowano klasyfikator ekspercki oraz meta-klasyfikator, które zostały napisane oraz przetestowane na prawdziwych zbiorach danych.

\bigskip
{\noindent\bf Słowa kluczowe:} klasyfikacja, meta-metody, meta-klasyfikator, niezrównoważone zbiory danych



\cleardoublepage
	\begin{center}
\large \bf
Improving classification with meta-algorithms
\end{center}

\section*{Abstract}
In the thesis are presented meta-algorithms of bagging, boosting and stacking and as well their influences on improving quality of data classification.  The methods were tested with authentic and variuos sets of data.
The thesis comprise methods of rating classifier to obtain reliable results and algorithms
used to balancing classes in sets with big advantage of the predominance of one class. Cooperation of the algorithms with their meta-algorithms were compared and tested. In the thesis an expert classifier and meta-classifier were proposed, which were written and tested on authentic data sets.

\bigskip
{\noindent\bf Keywords:} classification, meta-algorithms, meta-classifier, imbalanced data

\vfill
	\cleardoublepage
	\input{oswiadczenie.tex}
	
}
\begin{document}
\maketitle


\sloppy
\pagestyle{fancy}                       % Sets fancy header and footer
\fancyfoot{}                            % Delete current footer settings

\fancyhf{} % sets both header and footer to nothing
\renewcommand{\headrulewidth}{0pt}

\fancyfoot[LE,RO]{\thepage}  
\fancypagestyle{plain}{\pagestyle{fancy}}
\chapter{Wstęp}
napisac wstep
\chapter*{Cel pracy}
I jakiś cel pracy

\chapter{Wstęp teoretyczny}
\section{Klasyfikacja danych}
\todo{inline} może coś pisać o uczeniu maszynowym i że klasyfikacja jest nadzorowana?
Klasyfikacja jest to proces przyporządkowania danych do jednej z predefiniowanych klas na podstawie atrybutów tych danych. Algorytm klasyfikacji na podstawie analizy danych trenujących, zawierających atrybuty oraz klasę, tworzy model klasyfikacyjny. Stworzony model klasyfikacyjny wykorzystywany jest do predykcji klasy (kategorii) nowych danych bez określonej klasy. Celem algorytmu budującego model, jest  odnalezienie wzorców, w jaki sposób atrybuty obiektu wpływają na przynależność do danej klasy, starając się o to, aby wiedza na temat analizowanych danych była możliwie ogólna oraz niezależna od próby.

Klasyfikacja danych jest procesem dwuetapowym:
\begin{itemize}
	\item budowa modelu – proces ten polega na analizie obiektów z przyporządkowaną klasą oraz na budowie modelu opisującego predefiniowany zbiór klas danych,
	\item właściwa klasyfikacja – otrzymany model stosuje się do przydzielania klasy nowym obiektom.
\end{itemize}
Budowa modelu jest także procesem dwu-etapowym. Dzieli się ona na:
\begin{itemize}
	\item uczenie – klasyfikator budowany jest w oparciu o dane treningowe,
	\item ocena jakości klasyfikacji – jakoś klasyfikacji badana jest w oparciu o dane testowe.
\end{itemize}
W zależności od liczebności klas w zbiorze danych, możemy wyróżnić:
\begin{itemize}
	\item klasyfikację binarną – klasyfikator decyduje o przypisaniu obiektu do jednej z dwóch klas (np. czy człowiek jest zdrowy lub nie)
	\item klasyfikację wieloklasową – obiektowi przypisuje się jedną z wielu predefiniowanych klas.
\end{itemize}
Do reprezentacji danych uczących, testowych oraz do klasyfikacji najczęściej stosuje się system informacyjny.
\begin{table}[h]
\begin{center}
	\resizebox{\textwidth}{!}{%
	\begin{tabular}{|c|c|c|c|c|c|l|}
		
		\hline
		{\bf Zachmurzenie} & {\bf Temp.} & {\bf Temp. wody} & {\bf Opady} & {\bf Wiatr} & {\bf Pływać} & {\bf h(x)}\\
		\hline 
		słonecznie & 32 & 25 & brak & słaby & tak & tak \\
		\hline 
		słonecznie & 31 & 26 & brak & umiarkowany & tak & nie \\
		\hline
		pochmurnie & 22 & 15 & brak & b. mocny & nie & nie \\
		\hline
		pochmurnie & 20 & 18 & brak & słaby & tak & tak \\
		\hline
		całkowite zachmurzenie & 12 & 6 & brak & umiarkowany & nie & nie \\
		\hline
		całkowite zachmurzenie & 10 & 8 & duże & słaby & nie & nie \\
		\hline
		pochmurnie & 21 & 10 & brak & mocny & nie & tak \\
		\hline
		słonecznie & 25 & 17 & brak & umiarkowany & tak & nie \\
		\hline
		pochmurnie & 23 & 17 & przelotne & umiarkowany & nie & tak \\
		\hline
	\end{tabular}}
	\caption{Przykład danych treningowych składających się z 5 atrybutów oraz klasy decyzyjnej. W ostatniej kolumnie znajduje się wynik klasyfikacji. W pięciu przypadkach, klasyfikator poprawnie wskazał klasę.}
	\label{system_informacyjny}
\end{center}
\end{table}
\section{Przegląd algorytmów klasyfikacji danych}
\subsection{Drzewo decyzyjne}

\subsection{Naiwny klasyfikator bayesowski}

\subsection{Klasyfikator k najbliższych sąsiadów (kNN)}

\subsection{Las losowy}

\subsection{Bagging}

\subsection{Boosting}

\subsection{Stacking}

\section{Ocena poprawności klasyfikacji}

\subsection{Miary jakości klasyfikacji danych}
Jakość klasyfikacji można ocenić na podstawie kilku współczynników. Do ich obliczenia wykorzystuje się macierz pomyłek. Tworzona jest ona w oparciu o wynik klasyfikacji. Dla klasyfikacji binarnej macierz składa się z dwóch kolumn oraz dwóch wierszy. W wierszach znajdują się poprawne klasy decyzyjne, natomiast w kolumnach przewidziane przez klasyfikator. Zaklasyfikowane obiekty, umieszcza się w odpowiedniej grupie.
\begin{table}[h]
	\begin{center}
		\resizebox{\textwidth}{!}{%
		\begin{tabular}{ccc|c|c|c}

			&& \multicolumn{3}{ c }{Klasa predykowana} \\
			\cline{4-5}
			& \multicolumn{2}{ c| }{} & pozytywna & negatywna\\ \cline{3-5}
			\multicolumn{2}{ c| }{\multirow{2}{*}{\begin{tabular}[c]{@{}c@{}}Klasa\\rzeczywista\end{tabular}}} &pozytywna & prawdziwie pozytywna (TP) & fałszywie negatywna (FN)\\	\cline{3-5}
			\multicolumn{2}{ c| }{}&negatywna & fałszywie pozytywna (FP) & prawdziwie negatywna (TN)\\
			\cline{3-5}
			\end{tabular}}
		\caption{Macierz pomyłek}
		\label{macierz_pomylek}
		\end{center}
\end{table}

\chapter{Klasyfikacja danych niezrównoważonych}
\todo{wspomnieć o 2 podejścia -> preprecesing oraz algorytmy (algorytmamie się nie zajmuje)}

Większość istniejących algorytmów klasyfikacji, nastawiona jest na poprawną klasyfikację zbiorów o zrównoważonej liczebności wszystkich klas. Niestety, w rzeczywistych problemach, bardzo często zdarza się, że zbiory są mocno niezbilansowane.
\section{Dane niezrównoważone}
\todo{dodac odnosnik do one vs all}
Dane są niezrównoważone (ang. \textit{imbalanced data} jeśli klasy decyzyjne nie są przybliżeniu tak samo liczebne. Najmniejsza klasa, nazywana jest klasą mniejszościową (ang. \textit{minority class}), natomiast klasa dominująca, lub pozostałe połączone klasy (można połączyć pozostałe klasy w jedną, doprowadzając do klasyfikacji binarnej, one vs all), nazywana jest klasą większościową (ang. \textit{majority class}). W praktyce klasa mniejszościowa, zazwyczaj liczy około 10-20\% wszystkich przykładów. Często zdarzają się jednak takie problemy, gdzie to zróżnicowanie jest większe np.:
\begin{itemize}
	\item około 2\% transakcji kartami kredytowymi w GOCARDLESS to oszustwa \cite{gocardless}.
	
\end{itemize}
\todo{dodac przyklady danych mniejszosciowych}
W przytoczonych przykładach ważniejsza jest klasa mniejszościowa i wykrycie jej stanowi priorytet. Niezrównoważenie klas w zbiorze danych stanowi problem w fazie uczenia i znacząco obniża jakość klasyfikacji. Ze względu na częstość występowania klasy dominującej, klasyfikator preferuje tą klasę, dążąc do optymalizacji i obniżenia błędu error rate (\ref{error_rate}) nie biorąc pod uwagę rozłożenia klas w zbiorze. Klasyfikator może osiągnąć wysoką skuteczność klasyfikacji np. 95\% przy niskiej lub zerowej wykrywalności klasy mniejszościowej. 
Należy oczekiwać od klasyfikatora wysokiej skuteczności wykrywania klasy mniejszościowej, nawet kosztem pogorszenia rozpoznawania klasy większościowej.
Przykłady z klasy zdominowanej można podzielić na cztery grupy. K. Napierała i J. Stefanowski wyróżnili przykłady\cite{przykladyklas}:
\begin{itemize}
	\item safe - przykład bezpieczny, w jego sąsiedztwie zdecydowana większość obserwacji jest z tej samej klasy,
	\item borderline - graniczny, przykład niebezpieczny, w jego sąsiedztwie ilość przykładów z obu klas jest podobna
	\item outlier - poboczny, przykład niebezpieczny, w jego sąsiedztwie większość obserwacji jest z klasy przeciwnej, dominującej,
	\item rare - rzadki, przykład niebezpieczny, w jego sąsiedztwie występują tylko przykłady z klasy przeciwnej, większościowej.
\end{itemize}
\todo{dodac obrazek danyych safe border itd.}
\todo{podac wykresy przykłady danych niezrównoważonych}



\section{!!preprocessing danych niezrównoważonych}
W celu zrównoważenia rozkładu danych niezbilansowanych wprowadzono różne metody usuwania przykładów klasy dominującej lub tworzenia sztucznych obserwacji klasy mniejszościowej. Poniżej zostaną omówione metody, które zostały użyte podczas badań.
\subsection{Metody undersampling}
Jest to cała rodzina różnych metod, które usuwają przykłady z klasy większościowej. \textbf{Losowe usuwanie} (ang. \textit{random undersampling}), jak sama nazwa wskazuje, usuwa losowo przykłady z klasy dominującej. Rozwiązanie to ma niestety wadę. Jeśli usunie się zbyt dużo przykładów danego przypadku, można pozbawić klasyfikator bardzo ważnej informacji. \par
Lepszym rozwiązaniem jest świadome usuwanie przykładów spełniających określone kryteria. Taką metodą jest \textbf{undersampling z "Tomek links"}. Parę punktów Tomek link, definiuje się jako dwa punkty należące do różnych klas, z odległością równą $d(E_i,E_j)$, jeśli nie istnieje inny punkt $E_l$, taki, że $d(E_i,E_l) < d(E_i,E_j)$ lub $d(E_j,E_l) < d(E_i,E_j)$. Punkty tworzące Tomek link to szum lub punkt graniczny. Po znalezieniu takich punktów, usuwa się przykład z klasy dominującej. Usunięcie takiej obserwacji, powoduje rozszerzenie granicy klasy mniejszościowej. \par
\todo{opisac reszte te ktore wykorzystam}

\subsection{Metody oversampling}



\chapter{Przeprowadzone badania}
\section{Projekt klasyfikatora}

\section{Opis platformy i w jaki sposób zrealizowano badania}

\subsection{Język python}
Wszystkie badania i testy zostały napisane z wykorzystaniem języka python. Jest to język programowania interpretowany, wysokiego poziomu z dużą ilością dostępnych bibliotek. Python\cite{python} posiada dynamiczne zarządzanie typami oraz automatyczne zarządzanie pamięcią. Wspiera kilka paradygmatów programowania, takich ja: obiektowy, imperatywny, funkcyjny i proceduralny. Został zaprojektowany z myślą o czytelności kodu oraz składnią pozwalającą napisać program z mniejszą ilością kodu niż w językach C++ lub Java. Implementacje języka python dostępna jest na wiele systemów operacyjnych. Często wykorzystywany jest jako język skryptowy. Python jest projektem typu Open Source. \par
W pracy wykorzystano język python w wersji 2.7.11. Język ten wybrano ze względu na łatwość pisania w nim kodu, szybką możliwość nauki oraz na szeroki wachlarz dostępnych bibliotek. Ważnym argumentem w wyborze były gotowe biblioteki z klasyfikatorami oraz do pracy z klasyfikacją danych. Dostępność bibliotek do wizualizacji dla tego języka, pozwoliła na przedstawienie wyników testów w formie graficznej. Napisane testy, można w łatwy sposób rozbudować, zmodyfikować lub dodać nowe elementy.    

\subsection{Biblioteka scikit-learn}
Scikit-learn\cite{scikit} to proste i wydajne narzędzie do analizy i eksploracji danych. Jest to biblioteka uczenia maszynowego dla języka python. Rozpowszechnianie oparta jest na licencji BSD. W Scikt-learn zaimplementowane są (lub napisany jest kod obsługujący) różne algorytmy klasyfikacji, regresji, analizy skupień takie jak: maszyna wektorów nośnych, algorytmy najbliższego sąsiada, naiwny Bayes, drzewa decyzyjne, sieć neuronowa, zespoły klasyfikatorów. Z wykorzystaniem tej biblioteki można przygotować oraz przetworzyć odpowiednio dane. Możliwe jest także, ocenianie i wizualizacja wyników. \par
W testach użyto biblioteki scikit-learn w wersji 0.18.1. Wykorzystano z niej algorytmy klasyfikacji oraz wstępnego przetwarzania danych.
\subsection{Biblioteka imbalanced-learn}
Biblioteka imbalanced-learn\cite{imlearn} zawiera zestaw narzędzi do wstępnego przetwarzania danych niezrównoważonych. Posiada ona zaimplementowane różne metody under- oraz over-sampling do równoważenia zbiorów danych. W pracy wykorzystano te metody z biblioteki w wersji 0.2.1. 
\subsection{Pozostałe użyte biblioteki}
\subsubsection{Mlxtend}
Mlxtend (machine learning extensions)\cite{mlxtend} jest to biblioteka zawierająca różne narzędzia do pracy z danymi. W badaniach wykorzystano z niej algorytm Stacking.
\subsubsection{Numpy}
Numpy to pakiet umożliwiający obliczenia naukowe. Szczególnym elementem jest możliwość wykonywania obliczeń na tablicach N-wymiarowych. 
\subsubsection{Maptolib}
Maptolib to biblioteka pythona, która tworzy różnego rodzaju wykresy 2D oraz interaktywne na różnych platformach.
\subsubsection{Texttable}
Texttable to prosty moduł napisany w języku python, służący do produkcji prostych tabel ASCII. Został wykorzystany do prezentacji wyników w konsoli.
\subsubsection{Pylatex}
Pylatex to biblioteka pythona, służąca do tworzenia i kompilacji plików LaTeX. W pracy została wykorzystana do zapisu wyników w badań w plikach .tex oraz .pdf.
\subsection{opisac co zaimplementowane?}
\todo{ze wlasna krosswalidacja, ze wlasne miary, klasyfikator. itd}

\todo{inline} napisac o roznych F i o tym jak wyniki wyglada, pokazać test

\section{Opis danych użytych w badaniach}
Do przeprowadzenia badań użyto 26 różnych prawdziwych zbiorów danych (tabela \ref{danebadania}) pochodzących z repozytorium serwisu "The UCI Machine Learning Repository" \cite{uci}. Dane wybrano ze względu na różnorodność typów danych, ilości rekordów, atrybutów oraz zróżnicowanie rozkładu klas. Większość z danych była używana w publikacjach podobnych tematycznie\cite{hyper}\cite{StefImbalanced}. \par
Wszystkie dane zostały zapisane w skrypcie, w folderze $praca/data/files$, a opis szczegółowy danych znajduje się w folderze $praca/data/files/data\_descrytpion$. Do importu danych służą funkcje z pliku $praca/data/import\_data.py$. Do ogólnego importu danych z pliku, służy funkcja $importfile$, zaś wczytywanie danych użytych w projekcie odbywa się poprzez funkcje zaczynające się od $load\_$. Import danych z pliku odbywa się z wykorzystaniem funkcji z pakietu $numpy$ $genfromtext$ oraz $load\_txt$. Atrybuty posiadające dane kategoryczne zapisane w postaci łańcuchów znaków zostały zamienione na dane numeryczne. Cechy nominalne zostały zakodowane metodą $one hot encoding$. Dla danych zawierających więcej niż dwie klasy, klasa z najmniejszą liczebnością została wybrana jako klasa mniejszościowa, pozostałe klasy utworzyły klasę większościową. We wszystkich zbiorach danych, kategorie reprezentowane są w systemie binarnym. 5 zbiorów danych posiadało brakujące wartości. Zostały one zastąpione wartościami środkowymi zbioru (medianą).
\begin{table}[H]
	\begin{center}
		\resizebox{\textwidth}{!}{%
			\begin{tabular}{|c|c|c|c|c|c|}%
				\hline%
				Nazwa danych&L. el.&Atrybuty&Rozklad klas&\% kl. mn.&IR\\%
				\hline%
				abalone0\_4&4177&8&4103/74&1.77&55.45\\%
				abalone041629&4177&8&3842/335&8.02&11.47\\%
				abalone16\_29&4177&8&3916/261&6.25&15.0\\%
				balance\_scale&625&4&576/49&7.84&11.76\\%
				breast\_cancer&286&9&201/85&29.72&2.36\\%
				bupa&341&6&200/141&41.35&1.42\\%
				car&1728&6&1663/65&3.76&25.58\\%
				cmc&1473&9&1140/333&22.61&3.42\\%
				ecoli&336&7&301/35&10.42&8.6\\%
				german&1000&24&700/300&30.0&2.33\\%
				glass&214&9&197/17&7.94&11.59\\%
				haberman&306&3&225/81&26.47&2.78\\%
				heart\_cleveland&303&13&268/35&11.55&7.66\\%
				hepatitis&155&19&123/32&20.65&3.84\\%
				horse\_colic&368&22&232/136&36.96&1.71\\%
				ionosphere&351&34&225/126&35.9&1.79\\%
				new\_thyroid&215&5&185/30&13.95&6.17\\%
				postoperative&90&8&66/24&26.67&2.75\\%
				seeds&210&7&140/70&33.33&2.0\\%
				solar\_flare&1066&10&1023/43&4.03&23.79\\%
				transfusion&748&4&569/179&23.93&3.18\\%
				vehicle&846&18&647/199&23.52&3.25\\%
				vertebal&310&6&210/100&32.26&2.1\\%
				yeastME1&1484&8&1440/44&2.96&32.73\\%
				yeastME2&1484&8&1433/51&3.44&28.1\\%
				yeastME3&1484&8&1321/163&10.98&8.1\\%
				\hline%
			\end{tabular}}
			\caption{Dane użyte w badaniach wraz z charakterystyką.}
			\label{danebadania}
		\end{center}
	\end{table}

\section{Sposób mierzenia w sprawdzianie krzyżowym}
\section{Ocena klasyfikatora w sprawdzianie krzyżowym k-krotnym.}
W większości publikacji naukowych dotyczących klasyfikacji, ocena klasyfikatora mierzona jest z wykorzystaniem sprawdzianu krzyżowego (zwykle k=10) oraz przedstawionych wcześniej miar. Jednakże, w tych publikacjach nie został opisany sposób obliczania współczynników w czasie sprawdzianu krzyżowego. Wykorzystanie różnych sposobów, prowadzi do różnych wyników. Niektóre metody są mniej lub bardziej obciążone błędem. Różnice w wynikach, wynikające z przyjętej metody obliczeniowej, są szczególnie widoczne w sprawdzianie krzyżowym z losowym rozkładem danych oraz w klasyfikacji danych niezrównoważonych. Są dwie główne możliwości obliczania współczynników:
\begin{itemize}
	\item obliczanie wartości współczynników dla każdej k-iteracji (klasyfikatora), a następnie obliczenie średniej z tych iteracji
	\item stworzenie jednej wspólnej macierzy pomyłek dla każdej k-iteracji, a następnie obliczenie wskaźników.
\end{itemize}
W przypadku drugiego sposobu, poszczególne elementy macierzy pomyłek będą wynosić odpowiednio:
\[TP := \sum_{i=1}^{k} TP^{(i)}\]
\[FP := \sum_{i=1}^{k} FP^{(i)}\]
\[TN := \sum_{i=1}^{k} TN^{(i)}\]
\[FN := \sum_{i=1}^{k} FN^{(i)}\]
\subsection{Test sposobów oceny klasyfikatora}
W celu wyboru najlepszego sposobu oceny klasyfikatora, z najmniejszym błędem oraz wariancją wykonano pięć porównujących testów dla różnych metod obliczania miar. Wszystkie testy miały takie same założenia oraz sposób wykonania. Testy wykonano na wygenerowanych losowo danych dla różnej ilości przykładów pozytywnych (od 1\% do 10\%). Jakość klasyfikacji była oceniana dla równomiernego sprawdzianu krzyżowego oraz dla losowego. Symulacje odbyły się one w następujący sposób:

\begin{enumerate}
	\item Wygeneruj zbiór 1500 losowych próbek z 2 atrybutami, 2 klasami o rozkładzie 4:1
	\item Wykonaj niezrównoważenie zbioru z ratio 0.1
	\item Wybierz m=[1..10]*10 przykładów klasy mniejszościowej oraz 1000-m przykładów klasy dominującej
	\begin{enumerate}
		\item  Wykonaj N iteracji:
		\begin{enumerate}
			\item Wymieszaj dane
			\item Wykonaj sprawdzian krzyżowy k-krotny, k=10
			\item Oblicz współczynniki dla obu metod
		\end{enumerate}
		\item Oblicz odchylenie standardowe oraz średnie wartości współczynników
	\end{enumerate}
	\item Przedstaw wyniki odchylenia standardowego oraz średnie wartości współczynników dla różnego rozkładu klas.
\end{enumerate}
Wykonanie testu N-krotnie (najlepiej n>100000) pozwala na obliczenie "prawdziwych" wartości miar oceny klasyfikacji. Powtórzenie sprawdzianu krzyżowego wielokrotnie pozwala na ocenę błędu oraz wariancji miar dla każdej metodyki. Przeprowadzenie testu dla danych zawierających tylko 1\% obserwacji klasy mniejszościowej (przypadek ekstremalny, w zbiorze danych znajduje się wtedy tylko 10 takich przykładów) oznacza, że w niektórych iteracjach sprawdzianu krzyżowego nie będzie przykładów poprawnie sklasyfikowanych z tej klasy. Brak dobrze sklasyfikowanych przykładów klasy mniejszościowej może mieć także miejsce w losowym sprawdzianie krzyżowym. Wynika to z braku równomiernego rozkładu obu klas. 
\subsubsection{Dokładność oraz błąd klasyfikatora}
Dokładność klasyfikacji oraz błąd klasyfikatora, korzystając z metody pierwszej, będzie wynosić:
\[accuracy_{avg} := \frac{1}{k} \sum_{i=1}^{k} accuracy^{(i)}\]
a błąd klasyfikatora:
\[error\ rate_{avg} = 1 - accuracy_{avg}\]
Obliczając drugim sposobem, korzysta się z podstawowego wzoru z wykorzystaniem wspólnej macierzy pomyłek. \par
W przypadku dokładności oraz błędu klasyfikatora, niezależnie od przyjętej metodyki otrzymane wartości będą takie same, nieobciążone błędem.
\subsubsection{Czułość, specyficzność, FPR oraz precyzja}
Czułość, specyficzność, FPR oraz precyzja w metodzie pierwszej obliczą się wg. wzorów:
\[Sensitivity_{avg},\ Recall_{avg},\ TPR_{avg} := \sum_{i=1}^{k} TPR_{avg}^{(i)}\]
\[Specificity_{avg},\ TNR_{avg} := \sum_{i=1}^{k} TNR_{avg}^{(i)}\]
\[FPR_{avg} := \sum_{i=1}^{k} FPR_{avg}^{(i)}\]
\[Precision_{avg} := \sum_{i=1}^{k} Precision_{avg}^{(i)}\]
W drugim sposobie korzysta się z wspólnej macierzy pomyłek oraz z podstawowych wzorów. \par
Testy wyżej wymienionych współczynników, zostały przeprowadzone z wykorzystaniem skryptu $test\_wsk.py$. Zauważono, że w przypadku równomiernego sprawdzianu krzyżowego, różnica w wynikach jest bardzo mała, poniżej 0.5\%. Obie metody obarczone sa małym błędem i wariancją. Natomiast w przypadku sprawdzianu krzyżowego z rozkładem losowym, metoda druga okazała się lepsza. Obliczona czułość oraz precyzja sposobem drugim, uzyskały wyniki z mniejszym błędem, bliższe wartości "prawdziwej". Natomiast specyficzność w obu metodach wyszła taka sama, ze względu na dużą ilość przykładów z tej klasy. Zostało sprawdzone, że w momencie odwrócenia liczebności klas, specyficzność posiada taką samą charakterystykę jak czułość. Różnice w wynikach obu sposobów, zmniejszają się wraz ze wzrostem przykładów klasy większościowej. Zazwyczaj przy 10\% zawartości danych klasy zdominowanej w zbiorze, wyniki są takie same.
\todo{wstawic wykres i opisac}
\subsubsection{Miara $F_1$}
W niniejszej pracy oraz w dużej ilości publikacjach miara F-measure obliczana jest dla $\beta=1$, dlatego testy przeprowadzono tylko dla tej wartości. Miarę F, można obliczyć na trzy różne sposoby. Pierwszy polega na obliczeniu F dla każdego k klasyfikatora, a następnie uśrednienie wyników:
\[F_{avg} := \frac{1}{k} \sum_{i=1}^{k} F_1^{(i)}\]
Drugi sposób, to obliczenie średniej czułości i precyzji, a następnie miary F z podstawowego wzoru:
\[Pre_{avg} := \frac{1}{k} \sum_{i=1}^{k} Pre^{(i)}\]
\[Re_{avg} := \frac{1}{k} \sum_{i=1}^{k} Re^{(i)}\]
\[F_{pre, re} = 2 * \frac{Pre_{avg}*Re_{avg}}{Pre_{avg}+Re_{avg}} \]
Ostatnio sposób, to obliczenie współczynnika F ze wspólnej, końcowej macierzy pomyłek:
\[F_{tp, fp, fn} = \frac{2*TP}{2*TP+FP+FN} \]
Do powyższych wzorów można dodać jeszcze sposób odrzucający oceny klasyfikatorów, dla których precyzja lub czułość są niezdefiniowane. Ta metoda została odrzucona ze względu na zawyżanie końcowej oceny. \par 
Skrypt testujący powyższe trzy sposoby znajduje się w pliku $test\_f1.py$. Analizując otrzymane wyniki, zauważono, że najbardziej powtarzalne wyniki otrzymano korzystając z wzoru $F_{tp, fp, fn}$. W przypadku obu sprawdzianów krzyżowych, metoda ta generowała najmniejszy błąd. Zauważono, że niezależnie od ilości danych niezrównoważonych, otrzymane wyniki tą metodą różniły się nieznacznie, w przeciwieństwie do pozostałych sposobów. Podobnie jak w przypadku poprzednich miar, wraz ze wzrostem ilości danych niezrównoważonych, uzyskiwane wyniki były prawie takie same, niezależnie od sposobu obliczania.
\subsubsection{Miara G-mean}
Miara G-mean również może być obliczona na trzy różne sposoby. Pierwszy z nich to średnia z wszystkich klasyfikatorów:
\[G-mean_{avg} := \frac{1}{k} \sum_{i=1}^{k} G-mean^{(i)} \]
W drugim sposobie należy najpierw obliczyć średnią wartość czułości oraz specyficzności, a końcowy wynik G-mean oblicza się z głównego wzoru:
\[G-mean_{Se, Sp} = \sqrt{Sensitivity_{avg}*Specificity_{avg}} \]
W ostatniej metodzie obliczania G-mean, za czułość oraz specyficzność, wstawia się właściwie wzory, a wartość oblicza się na podstawie zsumowanej macierzy pomyłek.
\[G-mean_{tp, fp, fn} = \sqrt{\frac{TP}{TP + FN}*\frac{TN}{TN + FP}} \]
Skrypt testujący powyższe wzory znajduje się w pliku $test\_g\_mean.py$. Analizując wyniki równomiernego sprawdzianu krzyżowego, zaobserwowano, że  wyniki $G-mean_{tp, fp, fn}$ oraz $G-mean_{Se, Sp}$ pokrywają się. Natomiast w zwykłym sprawdzianie krzyżowym, pomiarem najmniej obarczonym błędem był $G-mean_{tp, fp, fn}$. Z wykorzystaniem tego wzoru, dla różnej zawartości klasy mniejszościowej w zbiorze danych otrzymano wyniki różniące się jedynie o kilka procent pomiędzy sobą, podczas gdy wyniki pozostałych metod różniły się aż o 15\%-30\%

\subsubsection{Krzywa ROC i miara AUC}
Wartość AUC w sprawdzianie krzyżowym można skonstruować na dwa sposoby. W pierwszej metodzie, konstruuje się krzywą ROC oraz oblicza się AUC dla każdego k klasyfikatora. Następnie oblicza się $AUC_{AVG}$ poprzez obliczenie średniej:
\[AUC_{AVG} := \frac{1}{k} \sum_{i=1}^{k} AUC^{(i)} \]
W przypadku sprawdzianu krzyżowego z losowym rozkładem klas, może okazać się, że nie sklasyfikowano żadnego przykładu pozytywnego. Wtedy skonstruowanie krzywej ROC oraz obliczenie AUC będzie niemożliwe. W takich przypadkach podczas obliczania $AUC_{AVG}$ można pominąć taki wynik. \par
Drugim sposobem jest połączenie prawdopodobieństwa przykładów testowych z każdej iteracji. Z połączonych obserwacji, konstruuje się jedną krzywą ROC i oblicza AUC. Korzystając z tego sposobu, zakłada się, że klasyfikator ma dobrze skalibrowane określanie prawdopodobieństwa. Test obu metod zostały przeprowadzone z użyciem skryptu $test_roc.py$.
\subsubsection{Podsumowanie}
Analizując przeprowadzone testy, najlepsze wyniki osiągnęły miary obliczone metodami opartymi na zsumowanej macierzy pomyłek oraz na łączeniu wyników testów z każdej iteracji sprawdzianu krzyżowego. Obliczone w ten sposób współczynniki miały najbardziej stabilne wyniki, najmniejszy błąd oraz wariancję. Poniżej przedstawiono tabelę (\ref{przykladoblwsp}) z obliczonymi współczynnikami na różne sposoby. W dalszej części pracy, wszystkie miary w sprawdzianie krzyżowym, będą obliczane nad podstawie wspólnej macierzy pomyłek.
\begin{table}[H]
	\begin{center}
		\resizebox{\textwidth}{!}{%
			\begin{tabular}{cccccccccccc}%
				\textbf{k-fold} & \textbf{Pos} & \textbf{Neg} & \textbf{TP} & \textbf{FP} & \textbf{FN} & \textbf{TN} & \textbf{Se} & \textbf{Sp} & \textbf{Pre} & \textbf{G-Mean} & $\mathbf{F_{1}}$\\
				\hline
				1 & 3 & 97 & 2 & 0 & 1 & 97 &  0,67 & 1,00 & 1,00 & 0,82 & 0,80\\
				2 & 3 & 97 & 0 & 0 & 3 & 97 &  0,00 & 1,00 & 0,00 & 0,00 & 0,00\\
				3 & 3 & 97 & 3 & 4 & 1 & 93 &  0,75 & 0,96 & 0,43 & 0,85 & 0,55\\

				  &   &    &   &   &   & \textbf{AVG}      & \textbf{0,47} & \textbf{0,99} & \textbf{0,48} & \textbf{0,55} & \textbf{0,45} \\
				  &   &    &   &   &   & \textbf{tp,fp,tn} & \textbf{0,50} & \textbf{0,99} & \textbf{0,56} & \textbf{0,70} & \textbf{0,53} \\
				  &   &    &   &   &   &  &  &  &  \multicolumn{3}{ c }{$\mathbf{G_{Se, Sp} =0,68}$ $\mathbf{F_{Pre, Re} =0,47}$}  \\
				

			\end{tabular}}%
			\caption{Przykład obliczonych miar dla równomiernego sprawdzianu krzyżowego. Dla k=2, gdzie nie było pozytywnie sklasyfikowanych przykładów, wartości sensitivity, precision, $F_1$ zostały ustawione na 0, aby uniknąć dzielenia przez zero. W wierszu oznaczonym jako "tp,fp,tn", wskaźniki zostały obliczone na podstawie wspólnej macierzy pomyłek.}
			\label{przykladoblwsp}
	\end{center}
\end{table}




\section{Sprawdzian krzyżowy, a oversampling}

\section{Porównanie klasyfikatorów}
\todo{przetestsowa wszystkie klasyfkatory, z metodami pre i bez}

Porównanie klasyfikatorów z domyślnymi ustawieniami, na takich samych typach danych. Jeżeli w tabelce występuje 0, to klasyfikator całkowicie błędnie klasyfikuje klasę mniejszościową.
\begin{table}[h]
	\begin{center}
		\resizebox{\textwidth}{!}{%
			\begin{tabular}{|c|c|c|c|c|c|c|c|c|}%
				\hline%
				&Drzewo&kNN&NKB&SVM&RForest&BAGGING&BOOSTING&STACKING\\%
				\hline%
				abalone0\_4&0.98&0.99&0.96&0.98&0.99&0.96&0.96&0.99\\%
				\hline%
				abalone0\_4\_16\_29&0.89&0.92&0.87&0.92&0.93&0.87&0.71&0.92\\%
				\hline%
				abalone16\_29&0.9&0.93&0.7&0.94&0.94&0.71&0.66&0.94\\%
				\hline%
				balance\_scale&0.86&0.91&0.92&0.92&0.91&0.92&0.92&0.92\\%
				\hline%
				breast\_cancer&0.62&0.64&0.73&0.66&0.71&0.72&0.37&0.7\\%
				\hline%
				bupa&0.65&0.64&0.53&0.58&0.71&0.55&0.52&0.64\\%
				\hline%
				car&0.99&0.97&0.88&0.98&0.98&0.89&0.98&0.99\\%
				\hline%
				cmc&0.7&0.75&0.68&0.77&0.76&0.68&0.63&0.77\\%
				\hline%
				ecoli&0.89&0.9&0.77&0.9&0.92&0.78&0.93&0.9\\%
				\hline%
				german&0.68&0.69&0.73&0.72&0.77&0.71&0.68&0.73\\%
				\hline%
				glass&0.89&0.9&0.45&0.92&0.92&0.49&0.85&0.92\\%
				\hline%
				haberman&0.58&0.71&0.75&0.71&0.67&0.76&0.63&0.75\\%
				\hline%
				heart\_cleveland&0.8&0.88&0.8&0.88&0.87&0.8&0.78&0.88\\%
				\hline%
				hepatitis&0.78&0.75&0.77&0.79&0.83&0.75&0.55&0.81\\%
				\hline%
				horse\_colic&0.79&0.72&0.79&0.68&0.86&0.79&0.61&0.82\\%
				\hline%
				ionosphere&0.87&0.83&0.88&0.94&0.93&0.89&0.78&0.92\\%
				\hline%
				new\_thyroid&0.97&0.97&0.97&0.86&0.98&0.97&0.97&0.97\\%
				\hline%
				postoperative&0.61&0.63&0.72&0.73&0.64&0.68&0.38&0.71\\%
				\hline%
				seeds&0.9&0.93&0.92&0.95&0.92&0.92&0.9&0.93\\%
				\hline%
				solar\_flare&0.94&0.95&0.67&0.96&0.94&0.67&0.52&0.96\\%
				\hline%
				transfusion&0.62&0.67&0.74&0.7&0.66&0.74&0.64&0.72\\%
				\hline%
				vehicle&0.94&0.93&0.66&0.77&0.97&0.67&0.75&0.93\\%
				\hline%
				vertebal&0.79&0.82&0.78&0.68&0.82&0.77&0.71&0.81\\%
				\hline%
				yeastME1&0.98&0.98&0.66&0.97&0.98&0.7&0.92&0.98\\%
				\hline%
				yeastME2&0.94&0.96&0.17&0.97&0.97&0.17&0.1&0.96\\%
				\hline%
				yeastME3&0.93&0.95&0.32&0.89&0.95&0.29&0.89&0.94\\%
				\hline%
			\end{tabular}}%
			\caption{Dokładność algorytmów}
			\label{tab1}
		\end{center}
	\end{table}
	
	\begin{table}[h]
		\begin{center}
			\resizebox{\textwidth}{!}{%
				\begin{tabular}{|c|c|c|c|c|c|c|c|c|}%
					\hline%
					&Drzewo&kNN&NKB&SVM&RForest&BAGGING&BOOSTING&STACKING\\%
					\hline%
					abalone0\_4&0.43&0.58&0.97&0.0&0.51&0.97&0.42&0.42\\%
					\hline%
					abalone0\_4\_16\_29&0.36&0.25&0.3&0.0&0.23&0.29&0.28&0.16\\%
					\hline%
					abalone16\_29&0.29&0.15&0.59&0.0&0.13&0.57&0.45&0.08\\%
					\hline%
					balance\_scale&0.08&0.0&0.0&0.0&0.0&0.0&0.0&0.0\\%
					\hline%
					breast\_cancer&0.41&0.2&0.45&0.07&0.38&0.44&0.6&0.24\\%
					\hline%
					bupa&0.6&0.46&0.77&0.0&0.52&0.78&0.48&0.45\\%
					\hline%
					car&0.92&0.62&1.0&0.77&0.71&1.0&0.82&0.92\\%
					\hline%
					cmc&0.34&0.29&0.6&0.15&0.28&0.6&0.35&0.15\\%
					\hline%
					ecoli&0.6&0.46&0.94&0.0&0.49&0.94&0.63&0.4\\%
					\hline%
					german&0.51&0.32&0.62&0.13&0.41&0.64&0.01&0.18\\%
					\hline%
					glass&0.41&0.12&0.76&0.0&0.06&0.71&0.12&0.0\\%
					\hline%
					haberman&0.32&0.35&0.21&0.01&0.25&0.25&0.2&0.17\\%
					\hline%
					heart\_cleveland&0.23&0.0&0.57&0.0&0.0&0.51&0.31&0.0\\%
					\hline%
					hepatitis&0.53&0.06&0.72&0.0&0.5&0.69&0.56&0.31\\%
					\hline%
					horse\_colic&0.73&0.57&0.76&0.25&0.76&0.76&0.56&0.64\\%
					\hline%
					ionosphere&0.8&0.56&0.74&0.84&0.87&0.76&0.73&0.83\\%
					\hline%
					new\_thyroid&0.9&0.77&0.87&0.0&0.9&0.87&0.8&0.83\\%
					\hline%
					postoperative&0.21&0.0&0.12&0.0&0.0&0.12&0.62&0.0\\%
					\hline%
					seeds&0.89&0.94&0.94&0.94&0.9&0.94&0.9&0.91\\%
					\hline%
					solar\_flare&0.14&0.0&0.93&0.0&0.07&0.93&0.56&0.0\\%
					\hline%
					transfusion&0.24&0.27&0.17&0.06&0.27&0.18&0.34&0.1\\%
					\hline%
					vehicle&0.85&0.88&0.83&0.01&0.95&0.83&0.74&0.81\\%
					\hline%
					vertebal&0.66&0.76&0.88&0.0&0.73&0.85&0.61&0.7\\%
					\hline%
					yeastME1&0.68&0.66&1.0&0.0&0.64&1.0&0.55&0.55\\%
					\hline%
					yeastME2&0.33&0.14&0.96&0.0&0.14&0.96&0.94&0.0\\%
					\hline%
					yeastME3&0.68&0.68&0.98&0.0&0.66&0.98&0.03&0.61\\%
					\hline%
				\end{tabular}}%
				\caption{Specyficzność algorytmów}
				\label{tab}
			\end{center}
		\end{table}
		
		\begin{table}[h]
			\begin{center}
				\resizebox{\textwidth}{!}{%
					\begin{tabular}{|c|c|c|c|c|c|c|c|c|}%
						\hline%
						&Drzewo&kNN&NKB&SVM&RForest&BAGGING&BOOSTING&STACKING\\%
						\hline%
						abalone0\_4&0.66&0.76&0.97&0.0&0.71&0.97&0.64&0.65\\%
						\hline%
						abalone0\_4\_16\_29&0.58&0.49&0.52&0.0&0.48&0.52&0.46&0.4\\%
						\hline%
						abalone16\_29&0.52&0.38&0.65&0.0&0.36&0.64&0.55&0.28\\%
						\hline%
						balance\_scale&0.27&0.0&0.0&0.0&0.0&0.0&0.0&0.0\\%
						\hline%
						breast\_cancer&0.54&0.41&0.61&0.25&0.57&0.6&0.41&0.46\\%
						\hline%
						bupa&0.64&0.59&0.53&0.0&0.66&0.56&0.52&0.59\\%
						\hline%
						car&0.96&0.78&0.94&0.87&0.84&0.94&0.9&0.96\\%
						\hline%
						cmc&0.52&0.51&0.65&0.38&0.5&0.65&0.5&0.38\\%
						\hline%
						ecoli&0.75&0.66&0.84&0.0&0.69&0.85&0.78&0.62\\%
						\hline%
						german&0.62&0.52&0.69&0.36&0.62&0.69&0.1&0.42\\%
						\hline%
						glass&0.62&0.34&0.57&0.0&0.24&0.58&0.33&0.0\\%
						\hline%
						haberman&0.47&0.54&0.45&0.11&0.45&0.48&0.4&0.41\\%
						\hline%
						heart\_cleveland&0.45&0.0&0.69&0.0&0.0&0.65&0.51&0.0\\%
						\hline%
						hepatitis&0.67&0.24&0.75&0.0&0.68&0.72&0.55&0.54\\%
						\hline%
						horse\_colic&0.77&0.68&0.78&0.48&0.84&0.78&0.6&0.77\\%
						\hline%
						ionosphere&0.85&0.74&0.84&0.91&0.92&0.86&0.77&0.89\\%
						\hline%
						new\_thyroid&0.94&0.88&0.93&0.0&0.94&0.93&0.89&0.91\\%
						\hline%
						postoperative&0.4&0.0&0.34&0.0&0.0&0.33&0.42&0.0\\%
						\hline%
						seeds&0.89&0.94&0.92&0.95&0.91&0.92&0.9&0.93\\%
						\hline%
						solar\_flare&0.37&0.0&0.78&0.0&0.26&0.78&0.54&0.0\\%
						\hline%
						transfusion&0.42&0.46&0.4&0.22&0.46&0.41&0.5&0.3\\%
						\hline%
						vehicle&0.91&0.91&0.71&0.1&0.97&0.72&0.75&0.88\\%
						\hline%
						vertebal&0.75&0.8&0.8&0.0&0.8&0.79&0.68&0.77\\%
						\hline%
						yeastME1&0.82&0.81&0.81&0.0&0.8&0.83&0.71&0.74\\%
						\hline%
						yeastME2&0.57&0.37&0.36&0.0&0.37&0.37&0.26&0.0\\%
						\hline%
						yeastME3&0.81&0.82&0.49&0.0&0.81&0.45&0.17&0.77\\%
						\hline%
					\end{tabular}}%
					\caption{G-mean}
					\label{tab3}
				\end{center}
			\end{table}
			
			\begin{table}[h]
				\begin{center}
					\resizebox{\textwidth}{!}{%
					\begin{tabular}{|c|c|c|c|c|c|c|c|c|}%
						\hline%
						&Drzewo&kNN&NKB&SVM&RForest&BAGGING&BOOSTING&STACKING\\%
						\hline%
						abalone0\_4&0.48&0.62&0.45&0.0&0.58&0.45&0.29&0.53\\%
						\hline%
						abalone0\_4\_16\_29&0.34&0.34&0.27&0.0&0.34&0.27&0.14&0.25\\%
						\hline%
						abalone16\_29&0.27&0.22&0.2&0.0&0.2&0.2&0.14&0.14\\%
						\hline%
						balance\_scale&0.08&0.0&0.0&0.0&0.0&0.0&0.0&0.0\\%
						\hline%
						breast\_cancer&0.39&0.25&0.49&0.11&0.44&0.48&0.36&0.32\\%
						\hline%
						bupa&0.58&0.51&0.58&0.0&0.6&0.59&0.45&0.51\\%
						\hline%
						car&0.86&0.61&0.39&0.76&0.72&0.4&0.79&0.85\\%
						\hline%
						cmc&0.34&0.35&0.46&0.23&0.34&0.46&0.3&0.23\\%
						\hline%
						ecoli&0.54&0.49&0.46&0.0&0.57&0.47&0.64&0.46\\%
						\hline%
						german&0.49&0.38&0.58&0.22&0.52&0.58&0.02&0.29\\%
						\hline%
						glass&0.37&0.16&0.18&0.0&0.11&0.18&0.11&0.0\\%
						\hline%
						haberman&0.29&0.39&0.31&0.02&0.28&0.35&0.22&0.27\\%
						\hline%
						heart\_cleveland&0.21&0.0&0.39&0.0&0.0&0.37&0.24&0.0\\%
						\hline%
						hepatitis&0.5&0.1&0.56&0.0&0.55&0.53&0.34&0.41\\%
						\hline%
						horse\_colic&0.71&0.6&0.73&0.37&0.8&0.73&0.51&0.72\\%
						\hline%
						ionosphere&0.81&0.7&0.82&0.91&0.91&0.83&0.7&0.88\\%
						\hline%
						new\_thyroid&0.89&0.87&0.9&0.0&0.92&0.9&0.87&0.89\\%
						\hline%
						postoperative&0.22&0.0&0.19&0.0&0.0&0.17&0.35&0.0\\%
						\hline%
						seeds&0.85&0.9&0.89&0.92&0.88&0.89&0.85&0.9\\%
						\hline%
						solar\_flare&0.15&0.0&0.19&0.0&0.09&0.18&0.09&0.0\\%
						\hline%
						transfusion&0.23&0.28&0.24&0.08&0.28&0.25&0.31&0.15\\%
						\hline%
						vehicle&0.87&0.86&0.54&0.02&0.94&0.54&0.58&0.84\\%
						\hline%
						vertebal&0.67&0.73&0.72&0.0&0.73&0.71&0.57&0.7\\%
						\hline%
						yeastME1&0.62&0.7&0.15&0.0&0.7&0.16&0.28&0.66\\%
						\hline%
						yeastME2&0.29&0.21&0.07&0.0&0.22&0.07&0.07&0.0\\%
						\hline%
						yeastME3&0.67&0.73&0.24&0.0&0.73&0.23&0.06&0.69\\%
						\hline%
					\end{tabular}}%
						\caption{F1}
						\label{tab4}
					\end{center}
				\end{table}

\setlength{\belowcaptionskip}{-20pt}
\chapter{Meta-metody}
\section{Bagging}
Klasyfikator bagging to meta-klasyfikator, który trenuje n klasyfikatorów bazowych na losowych podzbiorach oryginalnego zbioru danych, a następnie poprzez głosowanie lub uśrednianie indywidualnych prognoz nadaje ostateczną klasę. W procesie tworzenia klasyfikatora bagging, należy wybrać klasyfikator bazowy oraz określić liczbę n tworzonych instancji tego klasyfikatora. Można także zdefiniować czy tworzone losowe podzbiory mają być tworzone próbą boostrap, maksymalną liczebność podzbiorów oraz liczebność atrybutów. Zmieniając liczbę estymatorów, liczebność podzbiorów oraz atrybutów wpływa się na jakość klasyfikacji. Badania przeprowadzono z wykorzystaniem trzech różnych klasyfikatorów bazowych tj. z drzewem decyzyjnym, naiwnym klasyfikatorem bayesowskim oraz klasyfikatorem k najbliższych sąsiadów oraz dla różnych wartości przykładów oraz atrybutów. Każdy test został przeprowadzony z wykorzystaniem 5, 10, 15, 30, 50, 100 i 200 estymatorów. Wszystkie podzbiory zostały utworzone z wykorzystaniem próby boostrap, zatem podzbiory z taką samą liczebnością jak zbiór główny będą różniły się od siebie próbkami.
  
\subsection{Bagging z naiwnym klasyfikatorem bayesowskim}
Zbudowano klasyfikator bagging z naiwnym klasyfikatorem bazowym. Test ten znajduje się w pliku $bagging\_NB.py$. W pierwszym etapie wybrano standardowe ustawienia, czyli liczebność podzbiorów ($max\_samples$) oraz atrybutów ($max\_features$) była taka sama jak w zbiorze oryginalnym.
W tabelach powyżej przedstawiono dokładność klasyfikacji (tabela \ref{bagging_11}) oraz specyficzność klasy mniejszościowej (tabela \ref{bagging-specyficznosc11}). W kolumnie drugiej znajdują się wyniki dla samego naiwnego klasyfikatora Bayesa, natomiast w kolejnych kolumnach znajdują się wyniki meta-klasyfikatora bagging z różną ilością modeli bazowych (dla 5, 10, 15, 30, 50, 100, 200 klasyfikatorów). Analizując otrzymane wyniki, zauważono tylko minimalny wzrost (około 1\%) poprawy klasyfikacji obu klas dla połowy zbiorów danych. Dla prawie wszystkich zbiorów danych, wartość miary G-mean zmieniła się minimalnie, poniżej błędu. Test został wykonany wielokrotnie, a otrzymywane wyniki różniły się w bardzo niewielkim stopniu. 
\begin{table}[H]
	\tiny
	\begin{center}
		\resizebox{\textwidth}{!}{%
			\begin{tabular}{c|cccccccc}%
				Zbiór danych&NB&5&10&15&30&50&100&200\\%
				\hline%
				seeds&\textbf{0.9}&\textbf{0.9}&\textbf{0.9}&\textbf{0.9}&\textbf{0.9}&\textbf{0.9}&\textbf{0.9}&\textbf{0.9}\\%
				
				new\_thyroid&0.96&0.96&\textbf{0.97}&\textbf{0.97}&\textbf{0.97}&0.96&0.96&0.96\\%
				
				vehicle&0.66&\textbf{0.67}&\textbf{0.67}&\textbf{0.67}&\textbf{0.67}&0.66&0.66&0.66\\%
				
				ionosphere&\textbf{0.87}&0.85&0.85&0.85&0.86&0.86&\textbf{0.87}&\textbf{0.87}\\%
				
				vertebal&\textbf{0.78}&0.77&0.77&0.77&0.77&0.77&\textbf{0.78}&0.77\\%
				
				yeastME3&\textbf{0.27}&0.17&0.21&0.23&0.25&0.25&0.25&0.25\\%
				
				ecoli&0.78&0.77&0.79&\textbf{0.8}&0.79&0.79&0.78&0.79\\%
				
				bupa&0.54&0.53&\textbf{0.55}&0.54&\textbf{0.55}&\textbf{0.55}&0.54&0.54\\%
				
				horse\_colic&\textbf{0.78}&0.76&0.77&0.77&0.77&0.77&\textbf{0.78}&0.77\\%
				
				german&\textbf{0.73}&\textbf{0.73}&\textbf{0.73}&\textbf{0.73}&0.72&0.72&0.72&0.72\\%
				
				breast\_cancer&\textbf{0.72}&\textbf{0.72}&0.71&0.71&\textbf{0.72}&\textbf{0.72}&\textbf{0.72}&\textbf{0.72}\\%
				
				cmc&\textbf{0.68}&\textbf{0.68}&\textbf{0.68}&\textbf{0.68}&\textbf{0.68}&\textbf{0.68}&\textbf{0.68}&\textbf{0.68}\\%
				
				hepatitis&0.66&0.65&0.65&0.66&\textbf{0.68}&\textbf{0.68}&\textbf{0.68}&\textbf{0.68}\\%
				
				haberman&0.73&\textbf{0.74}&\textbf{0.74}&\textbf{0.74}&\textbf{0.74}&\textbf{0.74}&\textbf{0.74}&\textbf{0.74}\\%
				
				transfusion&\textbf{0.74}&\textbf{0.74}&\textbf{0.74}&\textbf{0.74}&\textbf{0.74}&\textbf{0.74}&\textbf{0.74}&\textbf{0.74}\\%
				
				car&0.89&\textbf{0.91}&0.9&0.9&0.9&0.9&0.9&0.9\\%
				
				glass&0.48&0.48&0.49&\textbf{0.53}&0.48&0.49&0.49&0.5\\%
				
				abalone16\_29&0.68&0.68&0.68&\textbf{0.69}&0.68&0.68&0.68&0.68\\%
				
				solar\_flare&\textbf{0.65}&0.62&0.61&0.63&0.62&0.62&0.62&0.63\\%
				
				heart\_cleveland&\textbf{0.81}&0.8&\textbf{0.81}&0.8&\textbf{0.81}&\textbf{0.81}&0.8&0.8\\%
				
				balance\_scale&\textbf{0.92}&\textbf{0.92}&\textbf{0.92}&\textbf{0.92}&\textbf{0.92}&\textbf{0.92}&\textbf{0.92}&\textbf{0.92}\\%
				
				postoperative&\textbf{0.67}&0.62&0.64&0.62&0.6&0.61&0.62&0.63\\%
				
			\end{tabular}}
			\caption{Dokładność klasyfikatora bagging, dla $max\_features = 1.0$ oraz $max\_samples = 1.0$}
			\label{bagging_11}
		\end{center}
	\end{table}
	\begin{table}[H]
		\tiny
		\begin{center}
			\resizebox{\textwidth}{!}{%
				\begin{tabular}{c|cccccccc}%
					Zbiór danych&NB&5&10&15&30&50&100&200\\%
					\hline%
					seeds&\textbf{0.91}&\textbf{0.91}&\textbf{0.91}&\textbf{0.91}&\textbf{0.91}&\textbf{0.91}&\textbf{0.91}&\textbf{0.91}\\%
					
					new\_thyroid&\textbf{0.87}&\textbf{0.87}&\textbf{0.87}&\textbf{0.87}&\textbf{0.87}&\textbf{0.87}&\textbf{0.87}&\textbf{0.87}\\%
					
					vehicle&0.84&0.83&0.84&0.84&\textbf{0.85}&0.84&0.84&0.84\\%
					
					ionosphere&0.76&0.76&\textbf{0.77}&\textbf{0.77}&\textbf{0.77}&0.76&0.76&0.76\\%
					
					vertebal&\textbf{0.87}&0.86&0.86&0.86&0.86&0.86&\textbf{0.87}&0.86\\%
					
					yeastME3&\textbf{0.99}&\textbf{0.99}&\textbf{0.99}&\textbf{0.99}&\textbf{0.99}&\textbf{0.99}&\textbf{0.99}&\textbf{0.99}\\%
					
					ecoli&\textbf{0.94}&\textbf{0.94}&\textbf{0.94}&\textbf{0.94}&\textbf{0.94}&\textbf{0.94}&\textbf{0.94}&\textbf{0.94}\\%
					
					bupa&0.74&0.75&\textbf{0.77}&\textbf{0.77}&0.74&0.75&0.75&0.74\\%
					
					horse\_colic&\textbf{0.75}&\textbf{0.75}&0.74&\textbf{0.75}&\textbf{0.75}&0.74&0.74&0.73\\%
					
					german&0.62&0.6&0.6&0.6&0.63&0.63&\textbf{0.65}&\textbf{0.65}\\%
					
					breast\_cancer&0.44&0.45&0.44&0.44&\textbf{0.46}&\textbf{0.46}&\textbf{0.46}&\textbf{0.46}\\%
					
					cmc&0.61&0.61&0.61&0.61&\textbf{0.62}&\textbf{0.62}&\textbf{0.62}&0.61\\%
					
					hepatitis&\textbf{0.78}&0.72&0.72&0.75&0.75&0.75&0.75&0.75\\%
					
					haberman&0.17&\textbf{0.2}&0.19&0.19&\textbf{0.2}&\textbf{0.2}&\textbf{0.2}&\textbf{0.2}\\%
					
					transfusion&0.2&\textbf{0.21}&\textbf{0.21}&\textbf{0.21}&\textbf{0.21}&\textbf{0.21}&\textbf{0.21}&\textbf{0.21}\\%
					
					car&\textbf{1.0}&\textbf{1.0}&\textbf{1.0}&\textbf{1.0}&\textbf{1.0}&\textbf{1.0}&\textbf{1.0}&\textbf{1.0}\\%
					
					glass&0.82&0.82&0.82&0.82&\textbf{0.88}&\textbf{0.88}&\textbf{0.88}&0.82\\%
					
					abalone16\_29&0.58&\textbf{0.59}&0.58&0.58&0.58&0.58&0.58&0.58\\%
					
					solar\_flare&\textbf{0.93}&\textbf{0.93}&\textbf{0.93}&\textbf{0.93}&\textbf{0.93}&\textbf{0.93}&\textbf{0.93}&\textbf{0.93}\\%
					
					heart\_cleveland&\textbf{0.63}&0.57&0.6&0.57&0.6&0.57&0.57&0.57\\%
					
					balance\_scale&\textbf{0.0}&\textbf{0.0}&\textbf{0.0}&\textbf{0.0}&\textbf{0.0}&\textbf{0.0}&\textbf{0.0}&\textbf{0.0}\\%
					
					postoperative&0.17&\textbf{0.21}&\textbf{0.21}&\textbf{0.21}&0.12&0.12&0.12&0.12\\%
					
				\end{tabular}}
				\caption{Specyficzność klasy mniejszościowej, dla klasyfikatora bagging i parametrów: $max\_features = 1.0$ oraz $max\_samples = 1.0$.}
				\label{bagging-specyficznosc11}
			\end{center}
		\end{table}
\par
W kolejnym teście znajdującym się w pliku $gridsearch/bagging\_NB.py$, wykonano zachłanne obliczenia polegające na wyłonieniu najlepszych ustawień klasyfikatora maksymalizującego miarę $F_1$ klasy mniejszościowej. W tym celu wykonano wyszukiwanie zachłanne najlepszych parametrów (liczby atrybutów oraz liczebności zbiorów) spośród $max\_features=[0.4, 0.6, 0.7, 0.8, 0.9, 1.0]$ oraz $max\_samples=[0.4, 0.6, 0.7, 0.8, 0.9, 1.0]$. Wyszukanie wykonano dla n-klasyfikatorów ze zbioru $[5, 10, 15, 30, 50, 100, 200]$ oraz dla każdego zbioru danych. Obliczona średnia wartość parametru $max\_features$ wyniosła 0.72, a $max\_samples$ 0.68, zaś mediana dla obu wartości to 0.7. \par
Zbudowany meta-klasyfikator bagging z parametrami $max\_features=0.72$ i $max\_samples=0,68$ poradził sobie lepiej z klasyfikacją niż klasyfikator bagging z domyślnymi parametrami oraz zwykły klasyfikator. W przypadku 17 zbiorów danych osiągnął lepszą dokładność (tabela \ref{bagging_acc2}). Dokładność klasyfikacji wzrosła średnio o 2-3\%, a w 3 zbiorach wzrost był zdecydowanie większy. Natomiast w 5 zbiorach danych, klasyfikator bagging uzyskał taką samą lub minimalne gorszą dokładność. W zespołach liczących powyżej 10 klasyfikatorów wystąpił minimalny przyrost dokładności poniżej 1\%. Prawie dla wszystkich danych, zwiększyła się czułość klasy większościowej, kosztem specyficzności (tabela \ref{baggin_spec2}) klasy mniejszościowej. Podobnie jak dla poprzedniego klasyfikatora, miara G-mean minimalnie spadła lub wzrosła w stosunku do NKB.
\begin{table}[H]
	\tiny
	\begin{center}
		\resizebox{\textwidth}{!}{%
			\begin{tabular}{c|cccccccc}%

				Zbiór danych&NB&5&10&15&30&50&100&200\\%
				\hline%
				seeds&0.9&0.89&0.9&\textbf{0.91}&\textbf{0.91}&\textbf{0.91}&0.9&0.9\\%
				new\_thyroid&0.96&\textbf{0.97}&\textbf{0.97}&\textbf{0.97}&\textbf{0.97}&\textbf{0.97}&\textbf{0.97}&\textbf{0.97}\\%
				vehicle&0.66&0.68&0.68&0.67&\textbf{0.69}&0.67&0.67&0.68\\%
				ionosphere&\textbf{0.87}&0.83&0.86&0.83&\textbf{0.87}&\textbf{0.87}&0.86&0.86\\%
				vertebal&\textbf{0.78}&0.75&0.76&0.77&0.77&0.77&0.77&\textbf{0.78}\\%
				yeastME3&0.27&\textbf{0.34}&0.25&0.17&0.21&0.23&0.23&0.22\\%
				ecoli&0.78&0.83&0.8&0.8&0.81&0.83&\textbf{0.84}&0.83\\%
				bupa&0.54&0.57&0.57&\textbf{0.6}&\textbf{0.6}&0.59&0.59&\textbf{0.6}\\%
				horse\_colic&0.78&0.73&0.76&0.73&0.76&0.77&0.78&\textbf{0.79}\\%
				german&\textbf{0.73}&0.66&0.66&0.68&0.7&0.72&0.72&\textbf{0.73}\\%
				breast\_cancer&0.72&\textbf{0.73}&\textbf{0.73}&\textbf{0.73}&\textbf{0.73}&\textbf{0.73}&\textbf{0.73}&\textbf{0.73}\\%
				cmc&0.68&\textbf{0.72}&0.71&0.71&0.71&0.71&\textbf{0.72}&\textbf{0.72}\\%
				hepatitis&0.66&0.63&0.67&\textbf{0.68}&0.66&0.66&\textbf{0.68}&0.67\\%
				haberman&0.73&0.74&\textbf{0.75}&\textbf{0.75}&0.74&0.74&0.74&0.74\\%
				transfusion&0.74&0.76&0.75&0.75&0.75&\textbf{0.77}&\textbf{0.77}&\textbf{0.77}\\%
				car&0.89&\textbf{0.92}&0.9&0.9&0.9&0.9&0.9&0.9\\%
				glass&0.48&\textbf{0.76}&0.59&0.61&0.59&0.6&0.6&0.59\\%
				abalone16\_29&0.68&0.8&\textbf{0.81}&0.8&0.79&0.79&0.79&0.79\\%
				solar\_flare&\textbf{0.65}&0.6&0.53&0.52&0.64&0.58&0.53&0.58\\%
				heart\_cleveland&0.81&0.78&0.8&0.81&0.82&0.82&\textbf{0.83}&0.82\\%
				balance\_scale&\textbf{0.92}&\textbf{0.92}&\textbf{0.92}&\textbf{0.92}&\textbf{0.92}&\textbf{0.92}&\textbf{0.92}&\textbf{0.92}\\%
				postoperative&0.67&\textbf{0.69}&0.63&0.62&0.64&0.62&0.63&0.64\\%
				\end{tabular}}
			\caption{Dokładność klasyfikatora bagging, dla $max\_features = 0.72$ oraz $max\_samples = 0.68$}
			\label{bagging_acc2}
		\end{center}
	\end{table}
\begin{table}[H]
		\tiny
	\begin{center}
		\resizebox{\textwidth}{!}{%
			\begin{tabular}{c|cccccccc}%

				Zbiór danych&NB&5&10&15&30&50&100&200\\%
				\hline%
				seeds&\textbf{0.91}&0.89&\textbf{0.91}&\textbf{0.91}&\textbf{0.91}&\textbf{0.91}&\textbf{0.91}&\textbf{0.91}\\%
				new\_thyroid&\textbf{0.87}&0.8&\textbf{0.87}&\textbf{0.87}&\textbf{0.87}&\textbf{0.87}&\textbf{0.87}&\textbf{0.87}\\%
				vehicle&0.84&\textbf{0.85}&0.82&0.83&0.82&0.82&0.82&0.83\\%
				ionosphere&0.76&\textbf{0.83}&0.75&0.79&0.75&0.73&0.72&0.74\\%
				vertebal&\textbf{0.87}&0.81&0.83&0.86&0.86&0.86&0.84&0.86\\%
				yeastME3&\textbf{0.99}&\textbf{0.99}&\textbf{0.99}&\textbf{0.99}&\textbf{0.99}&\textbf{0.99}&\textbf{0.99}&\textbf{0.99}\\%
				ecoli&\textbf{0.94}&0.83&0.86&0.8&0.91&0.91&0.91&0.91\\%
				bupa&\textbf{0.74}&0.5&0.63&0.62&0.64&0.7&0.67&0.67\\%
				horse\_colic&0.75&0.72&0.74&0.75&0.75&0.75&\textbf{0.76}&\textbf{0.76}\\%
				german&0.62&\textbf{0.73}&0.71&0.69&0.69&0.64&0.63&0.64\\%
				breast\_cancer&\textbf{0.44}&0.39&\textbf{0.44}&\textbf{0.44}&\textbf{0.44}&\textbf{0.44}&0.42&0.42\\%
				cmc&\textbf{0.61}&0.51&0.53&0.5&0.52&0.5&0.48&0.5\\%
				hepatitis&0.78&\textbf{0.84}&0.75&0.75&0.72&0.72&0.75&0.75\\%
				haberman&0.17&0.1&\textbf{0.19}&0.16&0.17&0.16&0.14&0.14\\%
				transfusion&\textbf{0.2}&0.15&0.16&0.17&0.15&0.16&0.16&0.16\\%
				car&\textbf{1.0}&0.45&\textbf{1.0}&\textbf{1.0}&\textbf{1.0}&\textbf{1.0}&\textbf{1.0}&\textbf{1.0}\\%
				glass&\textbf{0.82}&0.59&0.76&0.71&0.71&0.71&0.76&0.76\\%
				abalone16\_29&\textbf{0.58}&0.28&0.4&0.42&0.43&0.43&0.44&0.43\\%
				solar\_flare&\textbf{0.93}&0.81&\textbf{0.93}&\textbf{0.93}&\textbf{0.93}&\textbf{0.93}&\textbf{0.93}&\textbf{0.93}\\%
				heart\_cleveland&\textbf{0.63}&0.57&0.54&0.49&0.46&0.46&0.51&0.46\\%
				balance\_scale&\textbf{0.0}&\textbf{0.0}&\textbf{0.0}&\textbf{0.0}&\textbf{0.0}&\textbf{0.0}&\textbf{0.0}&\textbf{0.0}\\%
				postoperative&\textbf{0.17}&\textbf{0.17}&0.12&0.12&\textbf{0.17}&0.12&0.08&0.12\\%
			\end{tabular}}
			\caption{Specyficzność klasy mniejszościowej, dla klasyfikatora bagging i parametrów: $max\_features = 0.72$ oraz $max\_samples = 0.68$.}
			\label{baggin_spec2}
		\end{center}
	\end{table}
\subsection{Bagging drzewa decyzyjne}
Do kolejnych testów z meta-klasyfikatorem bagging wybrano drzewo decyzyjne. Budując klasyfikator drzewa decyzyjnego można zdefiniować maksymalną głębokość drzewa. Również meta-klasyfikator bagging można zbudować z drzew o różnej maksymalnej głębokości. Do testów wybrano drzewo bez ograniczenia głębokości oraz drzewa z ograniczeniami do maksymalnie 3, 5, 7, 10, 15 i 20 poziomu. Meta-klasyfikator był budowany z 5, 10, 20 lub 50 klasyfikatorów. Dla porównania, w wynikach zamieszczono pojedyncze drzewo decyzyjne o różnej głębokości. Przeprowadzony test znajduje się w pliku $bagging\_tree.py$. Ze względu na dużą objętość tabel, dla bazy $seeds$ i $new\_thyroid$, dokładność klasyfikacji i wykrywalność klas pozostała na takim samym poziomie niezależnie od głębokości drzewa i ilości klasyfikatorów. Natomiast w przypadku baz $vehicle$, $ionosphere$, $vertebal$, $yeastME3$, $solar\_flare$ klasyfikator bagging zwiększył dokładność klasyfikacji średnio o 2-3\%, czułość pozostała na niezmienionym poziomie, wzrosła natomiast specyficzność klasy mniejszościowej o 2-3\% oraz miara G-mean. Należy zaznaczyć, że dokładność oraz pozostałe współczynniki rosły wraz ze zwiększeniem ilości klasyfikatorów. Powyżej 50 klasyfikatorów, przyrost był minimalny. Dokładność klasyfikatora baggging dla pozostałych baz została przedstawiona w tabeli \ref{baggingdrzewoacc}. Zwiększając liczbę estymatorów wzrastała także dokładność, zwykle 2-3\% w stosunku do pojedynczego drzewa decyzyjnego. Zdecydowanie najlepsze wyniki w tej grupie oraz w reszcie baz zdanych, uzyskały klasyfikator z drzewem decyzyjnym z maksymalną głębokością równą 3. Wraz ze wzrostem dokładności, poprawiła się także czułość klasy większościowej, średnio o 2-10\%. Największy przyrost czułości nastąpił w klasyfikatorze bagging składającym się z 50 klasyfikatorów. Oprócz zbioru $breast\_cancer$, w którym specyficzność klasy zdominowanej wzrosła o 5\%, w pozostałych nastąpił wyraźny spadek rozpoznawalności klasy mniejszościowej. Zwiększając liczbę klasyfikatorów malał współczynnik specyficzności (tabela \ref{baggingdrzewospec}), a w przypadku bazy $glass$ spadł do zera. Zauważono, że wzrost wykrywalności obu klas nastąpił w bazach zawierających dużą liczbę przykładów bezpiecznych. W bazach trudniejszych do klasyfikacji, z dużą liczbą przykładów rzadkich oraz odstających, dokładność klasyfikacji klasy dominującej wzrosła kosztem wykrywalności klasy mniejszościowej. W czterech bazach, $glass$, $abalone16\_29$, $heart\_cleveland$, $postoperative$ wartość miary G-mean malała wraz ze zwiększeniem ilości klasyfikatorów, zaś w pozostałych bazach wzrastała. Dla baz $transfusion$, $balance\_scale$ oraz $car$ oprócz wzrostu jakości klasyfikacji klasy większościowej dla klasyfikatorów z drzewem decyzyjnym z maksymalna głębokością drzewa równą 3, nie stwierdzono różnic w pozostałych współczynnikach. W tabelach \ref{baggingdrzewoacc} i \ref{baggingdrzewospec} w kolumnach znajdują się klasyfikatory z różną maksymalną głębokością drzewa, a znak '-' oznacza drzewo bez ograniczenia.
\begin{table}[H]
	\tiny
	\begin{center}
		\resizebox{\textwidth}{!}{%
		\begin{tabular}{c|c|ccccccc}%
			\hline%
			Zbiór danych&Liczba est.&{-}&3&5&7&10&15&20\\%
			\hline%
			\multirow{5}{*}{breast\_cancer}&{-}&0.63&\textbf{0.73}&0.72&0.67&0.66&0.63&0.63\\%
			\cline{2%
				-%
				9}%
			&5&0.68&\textbf{0.7}&\textbf{0.7}&0.67&0.67&0.68&0.68\\%
			\cline{2%
				-%
				9}%
			&10&\textbf{0.71}&\textbf{0.71}&0.7&0.67&\textbf{0.71}&\textbf{0.71}&\textbf{0.71}\\%
			\cline{2%
				-%
				9}%
			&20&0.71&0.71&0.72&0.71&\textbf{0.74}&0.71&0.71\\%
			\cline{2%
				-%
				9}%
			&50&0.7&\textbf{0.72}&\textbf{0.72}&0.7&0.71&0.71&0.7\\%
			\hline%
			\multirow{5}{*}{cmc}&{-}&0.68&\textbf{0.78}&0.76&0.71&0.72&0.68&0.68\\%
			\cline{2%
				-%
				9}%
			&5&0.71&\textbf{0.78}&0.77&0.76&0.73&0.71&0.71\\%
			\cline{2%
				-%
				9}%
			&10&0.73&\textbf{0.77}&\textbf{0.77}&0.76&0.74&0.73&0.73\\%
			\cline{2%
				-%
				9}%
			&20&0.74&\textbf{0.78}&0.77&0.77&0.75&0.74&0.74\\%
			\cline{2%
				-%
				9}%
			&50&0.75&\textbf{0.78}&0.77&0.77&0.76&0.74&0.75\\%
			\hline%
			\multirow{5}{*}{hepatitis}&{-}&0.66&0.66&\textbf{0.68}&0.66&0.66&0.66&0.66\\%
			\cline{2%
				-%
				9}%
			&5&0.68&\textbf{0.7}&0.68&0.68&0.68&0.68&0.68\\%
			\cline{2%
				-%
				9}%
			&10&0.75&0.75&\textbf{0.77}&0.75&0.75&0.75&0.75\\%
			\cline{2%
				-%
				9}%
			&20&\textbf{0.72}&0.7&\textbf{0.72}&\textbf{0.72}&\textbf{0.72}&\textbf{0.72}&\textbf{0.72}\\%
			\cline{2%
				-%
				9}%
			&50&\textbf{0.7}&0.69&\textbf{0.7}&\textbf{0.7}&\textbf{0.7}&\textbf{0.7}&\textbf{0.7}\\%
			\hline%
			\multirow{5}{*}{haberman}&{-}&0.66&\textbf{0.75}&\textbf{0.75}&0.73&0.63&0.66&0.66\\%
			\cline{2%
				-%
				9}%
			&5&0.66&\textbf{0.74}&0.73&0.69&0.67&0.66&0.66\\%
			\cline{2%
				-%
				9}%
			&10&0.66&0.72&\textbf{0.74}&0.72&0.65&0.66&0.66\\%
			\cline{2%
				-%
				9}%
			&20&0.66&0.73&\textbf{0.74}&0.68&0.67&0.66&0.66\\%
			\cline{2%
				-%
				9}%
			&50&0.68&\textbf{0.74}&0.73&0.71&0.69&0.68&0.68\\%
			\hline%
			\multirow{5}{*}{glass}&{-}&0.7&\textbf{0.82}&0.75&0.69&0.7&0.7&0.7\\%
			\cline{2%
				-%
				9}%
			&5&0.73&\textbf{0.86}&0.78&0.73&0.73&0.73&0.73\\%
			\cline{2%
				-%
				9}%
			&10&0.84&\textbf{0.86}&0.83&0.84&0.84&0.84&0.84\\%
			\cline{2%
				-%
				9}%
			&20&0.86&\textbf{0.89}&0.87&0.86&0.86&0.86&0.86\\%
			\cline{2%
				-%
				9}%
			&50&0.86&\textbf{0.88}&0.87&0.86&0.86&0.86&0.86\\%
			\hline%
			\multirow{5}{*}{abalone16\_29}&{-}&0.91&\textbf{0.94}&0.93&0.93&0.91&0.91&0.91\\%
			\cline{2%
				-%
				9}%
			&5&0.93&\textbf{0.94}&\textbf{0.94}&0.93&0.93&0.93&0.93\\%
			\cline{2%
				-%
				9}%
			&10&0.93&\textbf{0.94}&\textbf{0.94}&0.93&0.93&0.93&0.93\\%
			\cline{2%
				-%
				9}%
			&20&\textbf{0.94}&\textbf{0.94}&\textbf{0.94}&\textbf{0.94}&0.93&\textbf{0.94}&\textbf{0.94}\\%
			\cline{2%
				-%
				9}%
			&50&0.93&\textbf{0.94}&\textbf{0.94}&\textbf{0.94}&\textbf{0.94}&\textbf{0.94}&\textbf{0.94}\\%
			\hline%
			\multirow{5}{*}{heart\_cleveland}&{-}&0.82&\textbf{0.86}&0.79&0.82&0.82&0.82&0.82\\%
			\cline{2%
				-%
				9}%
			&5&0.84&\textbf{0.87}&0.84&0.84&0.84&0.84&0.84\\%
			\cline{2%
				-%
				9}%
			&10&0.85&\textbf{0.86}&0.85&0.85&0.85&0.85&0.85\\%
			\cline{2%
				-%
				9}%
			&20&0.84&\textbf{0.88}&0.85&0.84&0.84&0.84&0.84\\%
			\cline{2%
				-%
				9}%
			&50&0.84&\textbf{0.87}&0.85&0.84&0.84&0.84&0.84\\%
			\hline%
			\multirow{5}{*}{postoperative}&{-}&0.67&\textbf{0.71}&0.7&0.69&0.7&0.67&0.67\\%
			\cline{2%
				-%
				9}%
			&5&0.68&\textbf{0.72}&0.68&0.69&0.7&0.68&0.68\\%
			\cline{2%
				-%
				9}%
			&10&0.7&\textbf{0.72}&0.7&0.69&0.7&0.7&0.7\\%
			\cline{2%
				-%
				9}%
			&20&0.7&\textbf{0.71}&0.7&0.68&\textbf{0.71}&0.7&0.7\\%
			\cline{2%
				-%
				9}%
			&50&0.66&\textbf{0.7}&0.69&0.67&0.64&0.66&0.66\\%
			\hline%
		\end{tabular}}
			\caption{Dokładność klasyfikatora bagging drzewo decyzyjne dla parametrów: $max\_features = 1$ oraz $max\_samples = 1$.}
			\label{baggingdrzewoacc}
		\end{center}
	\end{table}

\begin{table}[H]
	\tiny
	\begin{center}
		\resizebox{\textwidth}{!}{%
			\begin{tabular}{c|c|ccccccc}%
				\hline%
				Zbiór danych&Liczba est.&{-}&3&5&7&10&15&20\\%
				\hline%
				\multirow{5}{*}{breast\_cancer}&{-}&0.38&0.31&0.32&0.31&\textbf{0.4}&0.38&0.38\\%
				\cline{2%
					-%
					9}%
				&5&\textbf{0.39}&0.33&0.36&0.34&0.38&\textbf{0.39}&\textbf{0.39}\\%
				\cline{2%
					-%
					9}%
				&10&0.35&0.32&0.36&0.33&\textbf{0.38}&0.35&0.35\\%
				\cline{2%
					-%
					9}%
				&20&0.38&0.31&0.36&0.36&\textbf{0.45}&0.38&0.38\\%
				\cline{2%
					-%
					9}%
				&50&\textbf{0.4}&0.31&0.34&0.36&\textbf{0.4}&0.39&\textbf{0.4}\\%
				\hline%
				\multirow{5}{*}{cmc}&{-}&0.36&\textbf{0.39}&0.29&0.26&0.35&0.38&0.36\\%
				\cline{2%
					-%
					9}%
				&5&\textbf{0.3}&0.13&0.2&0.23&0.29&0.29&\textbf{0.3}\\%
				\cline{2%
					-%
					9}%
				&10&\textbf{0.27}&0.14&0.21&0.22&\textbf{0.27}&0.26&\textbf{0.27}\\%
				\cline{2%
					-%
					9}%
				&20&\textbf{0.29}&0.25&0.23&0.26&0.28&\textbf{0.29}&\textbf{0.29}\\%
				\cline{2%
					-%
					9}%
				&50&\textbf{0.32}&0.23&0.22&0.26&0.3&0.31&0.31\\%
				\hline%
				\multirow{5}{*}{hepatitis}&{-}&\textbf{0.59}&0.5&0.56&0.56&\textbf{0.59}&\textbf{0.59}&\textbf{0.59}\\%
				\cline{2%
					-%
					9}%
				&5&\textbf{0.56}&0.53&0.53&\textbf{0.56}&\textbf{0.56}&\textbf{0.56}&\textbf{0.56}\\%
				\cline{2%
					-%
					9}%
				&10&0.47&\textbf{0.53}&0.47&0.47&0.47&0.47&0.47\\%
				\cline{2%
					-%
					9}%
				&20&0.5&0.5&\textbf{0.53}&0.5&0.5&0.5&0.5\\%
				\cline{2%
					-%
					9}%
				&50&0.5&0.5&\textbf{0.53}&0.5&0.5&0.5&0.5\\%
				\hline%
				\multirow{5}{*}{haberman}&{-}&0.26&0.32&0.22&0.2&\textbf{0.33}&0.26&0.26\\%
				\cline{2%
					-%
					9}%
				&5&0.31&0.32&\textbf{0.38}&0.32&0.31&0.31&0.31\\%
				\cline{2%
					-%
					9}%
				&10&0.27&0.16&\textbf{0.36}&0.3&0.33&0.27&0.27\\%
				\cline{2%
					-%
					9}%
				&20&0.28&0.25&\textbf{0.36}&0.3&0.33&0.3&0.28\\%
				\cline{2%
					-%
					9}%
				&50&0.36&0.27&0.3&0.32&\textbf{0.37}&0.36&0.36\\%
				\hline%
				\multirow{5}{*}{glass}&{-}&\textbf{0.18}&0.12&0.12&0.12&\textbf{0.18}&\textbf{0.18}&\textbf{0.18}\\%
				\cline{2%
					-%
					9}%
				&5&\textbf{0.24}&0.0&\textbf{0.24}&\textbf{0.24}&\textbf{0.24}&\textbf{0.24}&\textbf{0.24}\\%
				\cline{2%
					-%
					9}%
				&10&0.12&0.0&\textbf{0.18}&0.12&0.12&0.12&0.12\\%
				\cline{2%
					-%
					9}%
				&20&0.0&0.0&\textbf{0.06}&0.0&0.0&0.0&0.0\\%
				\cline{2%
					-%
					9}%
				&50&0.0&0.0&0.0&0.0&0.0&0.0&0.0\\%
				\hline%
				\multirow{5}{*}{abalone16\_29}&{-}&\textbf{0.31}&0.09&0.11&0.23&0.28&\textbf{0.31}&\textbf{0.31}\\%
				\cline{2%
					-%
					9}%
				&5&\textbf{0.21}&0.07&0.12&0.14&0.2&\textbf{0.21}&\textbf{0.21}\\%
				\cline{2%
					-%
					9}%
				&10&0.17&0.06&0.13&0.14&\textbf{0.2}&0.18&0.17\\%
				\cline{2%
					-%
					9}%
				&20&0.18&0.09&0.12&0.13&\textbf{0.2}&\textbf{0.2}&0.19\\%
				\cline{2%
					-%
					9}%
				&50&0.17&0.08&0.1&0.15&\textbf{0.18}&\textbf{0.18}&\textbf{0.18}\\%
				\hline%
				\multirow{5}{*}{heart\_cleveland}&{-}&\textbf{0.17}&0.03&\textbf{0.17}&\textbf{0.17}&\textbf{0.17}&\textbf{0.17}&\textbf{0.17}\\%
				\cline{2%
					-%
					9}%
				&5&\textbf{0.2}&0.06&0.06&0.17&0.17&\textbf{0.2}&\textbf{0.2}\\%
				\cline{2%
					-%
					9}%
				&10&\textbf{0.11}&0.03&0.09&\textbf{0.11}&\textbf{0.11}&\textbf{0.11}&\textbf{0.11}\\%
				\cline{2%
					-%
					9}%
				&20&\textbf{0.09}&0.03&0.06&\textbf{0.09}&\textbf{0.09}&\textbf{0.09}&\textbf{0.09}\\%
				\cline{2%
					-%
					9}%
				&50&\textbf{0.06}&0.03&\textbf{0.06}&\textbf{0.06}&\textbf{0.06}&\textbf{0.06}&\textbf{0.06}\\%
				\hline%
				\multirow{5}{*}{postoperative}&{-}&0.17&0.08&0.12&\textbf{0.21}&0.17&0.17&0.17\\%
				\cline{2%
					-%
					9}%
				&5&0.12&0.12&0.12&\textbf{0.17}&\textbf{0.17}&0.12&0.12\\%
				\cline{2%
					-%
					9}%
				&10&\textbf{0.12}&\textbf{0.12}&0.08&0.08&\textbf{0.12}&\textbf{0.12}&\textbf{0.12}\\%
				\cline{2%
					-%
					9}%
				&20&0.21&0.12&0.17&0.17&\textbf{0.25}&0.21&0.21\\%
				\cline{2%
					-%
					9}%
				&50&\textbf{0.12}&0.04&\textbf{0.12}&\textbf{0.12}&\textbf{0.12}&\textbf{0.12}&\textbf{0.12}\\%
				\hline%
			\end{tabular}}
			\caption{Specyficzność klasy mniejszościowej, dla klasyfikatora bagging z drzewem decyzyjnym, z ustawionymi parametrami: $max\_features = 1$ oraz $max\_samples = 1$.}
			\label{baggingdrzewospec}
		\end{center}
	\end{table}
\par
W kolejnym teście klasyfikatora bagging z drzewem decyzyjnym (plik $gridsearch/bagging_tree.py$) obliczono najlepsze średnie ustawienia liczby atrybutów oraz liczby przykładów. Wyszukiwanie zachłanne wykonano dla parametrów $max\_features=[0.4, 0.6, 0.7, 0.8, 0.9, 1.0]$ oraz $max\_samples=[0.4, 0.6, 0.7, 0.8, 0.9, 1.0]$. Test przeprowadzono dla maksymalnej głębokości drzewa równej [None, 3, 5, 7, 10, 20], gdzie $none$ to bez ograniczenia głębokości oraz dla liczby klasyfikatorów równej [5, 10, 15, 20, 50, 100]. Dla każdej klasyfikatora wyłaniano najlepsze ustawienia, a nastepnie obliczono wartości średnie. Średnia liczba atrybutów wyniosła 0.85, średnia liczba przykładów 0.74, natomiast mediana wyniosła odpowiednio 0.9 oraz 0.8. Ponad połowa klasyfikatorów, najlepszą klasyfikację osiągnęła z wszystkimi atrybutami, a 20\% klasyfikatorów z wszystkimi przykładami. Kolejne obliczenia wykonano z parametrami $max\_features = 0.9$ oraz $max\_samples = 0.8$. W tabeli \ref{baggingdrzewoacc2} przedstawiono dokładność dla 6 zbiorów danych z największą liczbą przykładów bezpiecznych. Dokładność klasyfikacji poprawiła się głównie dla drzew z ograniczoną wysokością oraz z większa ilością klasyfikatorów. Podobnie jak poprzednio nastąpił wzrost specyficzności klasy mniejszościowej, miary F-1 klasy mniejszościowej (tabela \ref{baggingdrzewo2f1}) orazz miary G-mean. Natomias wykrywalność klasy większościowej wzrosła minimalnie. W tabeli \ref{baggingdrzewo2acc2} przedstawiono dokładność dla pozostałych danych. Otrzymane wyniki różniły się w granicy błędu od wyników uzyskanych z domyślnymi parametrami. W stosunku do normalnego drzewa decyzyjnego, nastąpił kilku procentowy wzrost dokładności. Meta-klasyfikator bagging najlepsze wyniki osiągał z drzewem decyzyjnym z maksymalną głębokością równą trzy. Zwiekszając ilość klasyfikatorów, rosła stoponiowo dokładność. Uzyskano podobne wartośći specyficzności klasy mniejszościowej (tabela \ref{baggingdrzewo2spec2}) oraz pozostałych współczynników jak z domyślnymi ustawieniami baggingu.
\begin{table}[H]
	\tiny
	\begin{center}
		\resizebox{\textwidth}{!}{%
			\begin{tabular}{c|c|ccccccc}%
				\hline%
				Zbiór danych&Liczba est.&{-}&3&5&7&10&15&20\\%
				\hline%
				\multirow{5}{*}{seeds}&{-}&\textbf{0.91}&\textbf{0.91}&\textbf{0.91}&\textbf{0.91}&\textbf{0.91}&\textbf{0.91}&\textbf{0.91}\\%
				\cline{2%
					-%
					9}%
				&5&0.89&\textbf{0.9}&0.89&0.89&0.89&0.89&0.89\\%
				\cline{2%
					-%
					9}%
				&10&\textbf{0.9}&\textbf{0.9}&\textbf{0.9}&\textbf{0.9}&\textbf{0.9}&\textbf{0.9}&\textbf{0.9}\\%
				\cline{2%
					-%
					9}%
				&20&\textbf{0.9}&\textbf{0.9}&\textbf{0.9}&\textbf{0.9}&\textbf{0.9}&\textbf{0.9}&\textbf{0.9}\\%
				\cline{2%
					-%
					9}%
				&50&\textbf{0.91}&0.9&\textbf{0.91}&\textbf{0.91}&\textbf{0.91}&\textbf{0.91}&\textbf{0.91}\\%
				\hline%
				\multirow{5}{*}{new\_thyroid}&{-}&\textbf{0.97}&\textbf{0.97}&\textbf{0.97}&\textbf{0.97}&\textbf{0.97}&\textbf{0.97}&\textbf{0.97}\\%
				\cline{2%
					-%
					9}%
				&5&\textbf{0.97}&\textbf{0.97}&\textbf{0.97}&\textbf{0.97}&\textbf{0.97}&\textbf{0.97}&\textbf{0.97}\\%
				\cline{2%
					-%
					9}%
				&10&\textbf{0.97}&\textbf{0.97}&\textbf{0.97}&\textbf{0.97}&\textbf{0.97}&\textbf{0.97}&\textbf{0.97}\\%
				\cline{2%
					-%
					9}%
				&20&\textbf{0.97}&\textbf{0.97}&\textbf{0.97}&\textbf{0.97}&\textbf{0.97}&\textbf{0.97}&\textbf{0.97}\\%
				\cline{2%
					-%
					9}%
				&50&\textbf{0.97}&\textbf{0.97}&\textbf{0.97}&\textbf{0.97}&\textbf{0.97}&\textbf{0.97}&\textbf{0.97}\\%
				\hline%
				\multirow{5}{*}{vehicle}&{-}&0.93&0.9&0.93&\textbf{0.94}&\textbf{0.94}&0.93&0.93\\%
				\cline{2%
					-%
					9}%
				&5&\textbf{0.95}&0.92&0.93&0.94&\textbf{0.95}&\textbf{0.95}&\textbf{0.95}\\%
				\cline{2%
					-%
					9}%
				&10&\textbf{0.96}&0.93&\textbf{0.96}&\textbf{0.96}&\textbf{0.96}&\textbf{0.96}&\textbf{0.96}\\%
				\cline{2%
					-%
					9}%
				&20&\textbf{0.96}&0.93&\textbf{0.96}&\textbf{0.96}&\textbf{0.96}&\textbf{0.96}&\textbf{0.96}\\%
				\cline{2%
					-%
					9}%
				&50&\textbf{0.97}&0.94&0.95&0.96&\textbf{0.97}&\textbf{0.97}&\textbf{0.97}\\%
				\hline%
				\multirow{5}{*}{ionosphere}&{-}&0.86&0.86&\textbf{0.87}&\textbf{0.87}&0.86&0.86&0.86\\%
				\cline{2%
					-%
					9}%
				&5&\textbf{0.9}&0.89&0.89&0.89&\textbf{0.9}&\textbf{0.9}&\textbf{0.9}\\%
				\cline{2%
					-%
					9}%
				&10&\textbf{0.91}&0.9&0.9&\textbf{0.91}&\textbf{0.91}&\textbf{0.91}&\textbf{0.91}\\%
				\cline{2%
					-%
					9}%
				&20&\textbf{0.91}&\textbf{0.91}&\textbf{0.91}&\textbf{0.91}&\textbf{0.91}&\textbf{0.91}&\textbf{0.91}\\%
				\cline{2%
					-%
					9}%
				&50&0.91&0.91&0.91&\textbf{0.92}&0.91&0.91&0.91\\%
				\hline%
				\multirow{5}{*}{vertebal}&{-}&0.71&0.71&0.72&\textbf{0.73}&0.71&0.71&0.71\\%
				\cline{2%
					-%
					9}%
				&5&0.71&\textbf{0.72}&0.71&0.71&0.71&0.71&0.71\\%
				\cline{2%
					-%
					9}%
				&10&0.71&0.71&0.71&\textbf{0.72}&0.71&0.71&0.71\\%
				\cline{2%
					-%
					9}%
				&20&0.71&0.71&\textbf{0.72}&\textbf{0.72}&\textbf{0.72}&0.71&0.71\\%
				\cline{2%
					-%
					9}%
				&50&\textbf{0.73}&0.72&\textbf{0.73}&\textbf{0.73}&\textbf{0.73}&\textbf{0.73}&\textbf{0.73}\\%
				\hline%
				\multirow{5}{*}{yeastME3}&{-}&0.93&\textbf{0.94}&\textbf{0.94}&\textbf{0.94}&0.93&0.93&0.93\\%
				\cline{2%
					-%
					9}%
				&5&0.94&\textbf{0.95}&\textbf{0.95}&0.94&0.94&0.94&0.94\\%
				\cline{2%
					-%
					9}%
				&10&0.94&\textbf{0.95}&\textbf{0.95}&0.94&0.94&0.94&0.94\\%
				\cline{2%
					-%
					9}%
				&20&0.94&\textbf{0.95}&0.94&0.94&\textbf{0.95}&0.94&0.94\\%
				\cline{2%
					-%
					9}%
				&50&\textbf{0.95}&\textbf{0.95}&0.94&0.94&\textbf{0.95}&\textbf{0.95}&\textbf{0.95}\\%
				\hline%
			\end{tabular}}
			\caption{Dokładność klasyfikatora bagging, dla $max\_features = 0.9$ oraz $max\_samples = 0.8$}
			\label{baggingdrzewoacc2}
		\end{center}
	\end{table}
	
\begin{table}[H]
	\tiny
	\begin{center}
		\resizebox{\textwidth}{!}{%
			\begin{tabular}{c|c|ccccccc}%
				\hline%
				Zbiór danych&Liczba est.&{-}&3&5&7&10&15&20\\%
				\hline%
				\multirow{5}{*}{seeds}&{-}&\textbf{0.87}&\textbf{0.87}&\textbf{0.87}&\textbf{0.87}&\textbf{0.87}&\textbf{0.87}&\textbf{0.87}\\%
				\cline{2%
					-%
					9}%
				&5&0.83&\textbf{0.85}&0.83&0.83&0.83&0.83&0.83\\%
				\cline{2%
					-%
					9}%
				&10&\textbf{0.85}&\textbf{0.85}&\textbf{0.85}&\textbf{0.85}&\textbf{0.85}&\textbf{0.85}&\textbf{0.85}\\%
				\cline{2%
					-%
					9}%
				&20&\textbf{0.85}&0.84&\textbf{0.85}&\textbf{0.85}&\textbf{0.85}&\textbf{0.85}&\textbf{0.85}\\%
				\cline{2%
					-%
					9}%
				&50&\textbf{0.87}&0.86&\textbf{0.87}&\textbf{0.87}&\textbf{0.87}&\textbf{0.87}&\textbf{0.87}\\%
				\hline%
				\multirow{5}{*}{new\_thyroid}&{-}&\textbf{0.88}&\textbf{0.88}&\textbf{0.88}&\textbf{0.88}&\textbf{0.88}&\textbf{0.88}&\textbf{0.88}\\%
				\cline{2%
					-%
					9}%
				&5&\textbf{0.9}&\textbf{0.9}&\textbf{0.9}&\textbf{0.9}&\textbf{0.9}&\textbf{0.9}&\textbf{0.9}\\%
				\cline{2%
					-%
					9}%
				&10&\textbf{0.88}&\textbf{0.88}&\textbf{0.88}&\textbf{0.88}&\textbf{0.88}&\textbf{0.88}&\textbf{0.88}\\%
				\cline{2%
					-%
					9}%
				&20&\textbf{0.88}&\textbf{0.88}&\textbf{0.88}&\textbf{0.88}&\textbf{0.88}&\textbf{0.88}&\textbf{0.88}\\%
				\cline{2%
					-%
					9}%
				&50&\textbf{0.9}&0.88&\textbf{0.9}&\textbf{0.9}&\textbf{0.9}&\textbf{0.9}&\textbf{0.9}\\%
				\hline%
				\multirow{5}{*}{vehicle}&{-}&0.86&0.81&0.84&\textbf{0.87}&\textbf{0.87}&0.86&0.86\\%
				\cline{2%
					-%
					9}%
				&5&\textbf{0.88}&0.82&0.85&0.87&\textbf{0.88}&\textbf{0.88}&\textbf{0.88}\\%
				\cline{2%
					-%
					9}%
				&10&\textbf{0.91}&0.85&\textbf{0.91}&\textbf{0.91}&\textbf{0.91}&\textbf{0.91}&\textbf{0.91}\\%
				\cline{2%
					-%
					9}%
				&20&\textbf{0.92}&0.85&0.91&0.91&\textbf{0.92}&\textbf{0.92}&\textbf{0.92}\\%
				\cline{2%
					-%
					9}%
				&50&\textbf{0.93}&0.87&0.89&0.92&\textbf{0.93}&\textbf{0.93}&\textbf{0.93}\\%
				\hline%
				\multirow{5}{*}{ionosphere}&{-}&0.81&0.79&\textbf{0.82}&\textbf{0.82}&0.81&0.81&0.81\\%
				\cline{2%
					-%
					9}%
				&5&\textbf{0.86}&0.84&0.83&0.84&\textbf{0.86}&\textbf{0.86}&\textbf{0.86}\\%
				\cline{2%
					-%
					9}%
				&10&0.86&0.85&0.85&\textbf{0.87}&0.86&0.86&0.86\\%
				\cline{2%
					-%
					9}%
				&20&0.87&0.87&\textbf{0.88}&0.87&0.87&0.87&0.87\\%
				\cline{2%
					-%
					9}%
				&50&0.87&0.87&\textbf{0.88}&\textbf{0.88}&0.87&0.87&0.87\\%
				\hline%
				\multirow{5}{*}{vertebal}&{-}&0.63&0.62&0.64&\textbf{0.66}&0.63&0.63&0.63\\%
				\cline{2%
					-%
					9}%
				&5&0.62&\textbf{0.63}&0.62&0.62&0.62&0.62&0.62\\%
				\cline{2%
					-%
					9}%
				&10&0.61&0.61&\textbf{0.62}&\textbf{0.62}&0.61&0.61&0.61\\%
				\cline{2%
					-%
					9}%
				&20&0.62&0.62&\textbf{0.63}&\textbf{0.63}&\textbf{0.63}&0.62&0.62\\%
				\cline{2%
					-%
					9}%
				&50&\textbf{0.65}&0.63&0.64&\textbf{0.65}&\textbf{0.65}&\textbf{0.65}&\textbf{0.65}\\%
				\hline%
				\multirow{5}{*}{yeastME3}&{-}&0.68&\textbf{0.74}&0.72&0.72&0.69&0.68&0.68\\%
				\cline{2%
					-%
					9}%
				&5&0.75&\textbf{0.78}&0.76&0.74&0.74&0.75&0.75\\%
				\cline{2%
					-%
					9}%
				&10&0.72&\textbf{0.76}&0.74&0.73&0.72&0.73&0.72\\%
				\cline{2%
					-%
					9}%
				&20&0.73&\textbf{0.76}&0.73&0.74&0.74&0.73&0.73\\%
				\cline{2%
					-%
					9}%
				&50&0.74&0.74&0.74&0.73&0.74&\textbf{0.75}&0.74\\%
				\hline%
			\end{tabular}}
			\caption{Miara F-1 klasy mniejszościowej. Klasyfikator bagging drzewo decyzyjne z parametrami $max\_features = 0.9$ oraz $max\_samples = 0.8$.}
			\label{baggingdrzewo2f1}
		\end{center}
	\end{table}
	
\begin{table}[H]
	\tiny
	\begin{center}
		\resizebox{\textwidth}{!}{%
			\begin{tabular}{c|c|ccccccc}%
				\hline%
				Zbiór danych&Liczba est.&{-}&3&5&7&10&15&20\\%
				\hline%
				\multirow{5}{*}{breast\_cancer}&{-}&0.63&\textbf{0.73}&\textbf{0.73}&0.71&0.66&0.63&0.63\\%
				\cline{2%
					-%
					9}%
				&5&0.65&\textbf{0.74}&0.7&0.69&0.67&0.66&0.65\\%
				\cline{2%
					-%
					9}%
				&10&0.71&\textbf{0.74}&0.72&0.71&0.71&0.71&0.71\\%
				\cline{2%
					-%
					9}%
				&20&0.67&\textbf{0.71}&\textbf{0.71}&0.69&0.68&0.68&0.67\\%
				\cline{2%
					-%
					9}%
				&50&0.69&\textbf{0.73}&0.7&0.71&0.69&0.69&0.69\\%
				\hline%
				\multirow{5}{*}{cmc}&{-}&0.68&\textbf{0.78}&0.76&0.71&0.71&0.68&0.68\\%
				\cline{2%
					-%
					9}%
				&5&0.72&\textbf{0.78}&0.77&0.75&0.74&0.72&0.72\\%
				\cline{2%
					-%
					9}%
				&10&0.74&\textbf{0.78}&\textbf{0.78}&0.76&0.74&0.74&0.74\\%
				\cline{2%
					-%
					9}%
				&20&0.75&\textbf{0.78}&0.77&0.77&0.75&0.74&0.75\\%
				\cline{2%
					-%
					9}%
				&50&0.74&0.77&\textbf{0.78}&0.77&0.76&0.75&0.75\\%
				\hline%
				\multirow{5}{*}{hepatitis}&{-}&0.7&0.66&\textbf{0.72}&0.7&0.7&0.7&0.7\\%
				\cline{2%
					-%
					9}%
				&5&0.64&\textbf{0.74}&0.65&0.64&0.64&0.64&0.64\\%
				\cline{2%
					-%
					9}%
				&10&0.71&\textbf{0.72}&\textbf{0.72}&0.7&0.71&0.71&0.71\\%
				\cline{2%
					-%
					9}%
				&20&0.72&0.72&\textbf{0.73}&0.72&0.72&0.72&0.72\\%
				\cline{2%
					-%
					9}%
				&50&\textbf{0.7}&0.68&0.69&\textbf{0.7}&\textbf{0.7}&\textbf{0.7}&\textbf{0.7}\\%
				\hline%
				\multirow{5}{*}{haberman}&{-}&0.66&\textbf{0.75}&\textbf{0.75}&0.72&0.67&0.64&0.66\\%
				\cline{2%
					-%
					9}%
				&5&0.68&\textbf{0.75}&0.73&0.71&0.68&0.68&0.68\\%
				\cline{2%
					-%
					9}%
				&10&0.7&\textbf{0.75}&\textbf{0.75}&0.73&0.71&0.7&0.7\\%
				\cline{2%
					-%
					9}%
				&20&0.68&\textbf{0.75}&\textbf{0.75}&0.69&0.67&0.68&0.68\\%
				\cline{2%
					-%
					9}%
				&50&0.67&\textbf{0.75}&0.74&0.71&0.67&0.67&0.67\\%
				\hline%
				\multirow{5}{*}{glass}&{-}&0.75&\textbf{0.82}&0.76&0.74&0.75&0.75&0.75\\%
				\cline{2%
					-%
					9}%
				&5&0.86&\textbf{0.9}&0.86&0.86&0.86&0.86&0.86\\%
				\cline{2%
					-%
					9}%
				&10&0.87&\textbf{0.9}&0.86&0.87&0.87&0.87&0.87\\%
				\cline{2%
					-%
					9}%
				&20&0.85&\textbf{0.88}&0.8&0.85&0.85&0.85&0.85\\%
				\cline{2%
					-%
					9}%
				&50&0.86&\textbf{0.89}&0.84&0.86&0.86&0.86&0.86\\%
				\hline%
				\multirow{5}{*}{abalone16\_29}&{-}&0.91&\textbf{0.94}&0.93&0.93&0.91&0.91&0.91\\%
				\cline{2%
					-%
					9}%
				&5&0.93&\textbf{0.94}&\textbf{0.94}&\textbf{0.94}&0.93&0.93&0.93\\%
				\cline{2%
					-%
					9}%
				&10&0.93&\textbf{0.94}&\textbf{0.94}&\textbf{0.94}&\textbf{0.94}&\textbf{0.94}&0.93\\%
				\cline{2%
					-%
					9}%
				&20&\textbf{0.94}&\textbf{0.94}&\textbf{0.94}&\textbf{0.94}&\textbf{0.94}&\textbf{0.94}&\textbf{0.94}\\%
				\cline{2%
					-%
					9}%
				&50&0.93&\textbf{0.94}&\textbf{0.94}&\textbf{0.94}&0.93&0.93&0.93\\%
				\hline%
				\multirow{5}{*}{heart\_cleveland}&{-}&0.8&\textbf{0.86}&0.8&0.8&0.8&0.8&0.8\\%
				\cline{2%
					-%
					9}%
				&5&\textbf{0.85}&0.83&0.84&0.84&\textbf{0.85}&\textbf{0.85}&\textbf{0.85}\\%
				\cline{2%
					-%
					9}%
				&10&\textbf{0.86}&0.85&0.84&\textbf{0.86}&\textbf{0.86}&\textbf{0.86}&\textbf{0.86}\\%
				\cline{2%
					-%
					9}%
				&20&0.86&\textbf{0.87}&0.85&0.86&0.86&0.86&0.86\\%
				\cline{2%
					-%
					9}%
				&50&0.84&\textbf{0.87}&0.85&0.84&0.84&0.84&0.84\\%
				\hline%
				\multirow{5}{*}{postoperative}&{-}&0.66&\textbf{0.73}&0.7&0.68&0.63&0.66&0.66\\%
				\cline{2%
					-%
					9}%
				&5&0.64&\textbf{0.68}&0.62&0.64&0.64&0.64&0.64\\%
				\cline{2%
					-%
					9}%
				&10&\textbf{0.67}&\textbf{0.67}&0.63&0.66&\textbf{0.67}&\textbf{0.67}&\textbf{0.67}\\%
				\cline{2%
					-%
					9}%
				&20&0.64&\textbf{0.68}&0.66&0.64&0.64&0.64&0.64\\%
				\cline{2%
					-%
					9}%
				&50&0.63&\textbf{0.68}&0.66&0.66&0.64&0.63&0.63\\%
				\hline%
			\end{tabular}}
			\caption{Dokładność klasyfikatora bagging drzewo decyzyjne, dla $max\_features = 0.9$ oraz $max\_samples = 0.8$}
			\label{baggingdrzewo2acc2}
		\end{center}
	\end{table}	
\begin{table}[H]
	\tiny
	\begin{center}
		\resizebox{\textwidth}{!}{%
			\begin{tabular}{c|c|ccccccc}%
				\hline%
				Zbiór danych&Liczba est.&{-}&3&5&7&10&15&20\\%
				\hline%
				\multirow{5}{*}{breast\_cancer}&{-}&\textbf{0.39}&0.31&0.34&0.32&\textbf{0.39}&\textbf{0.39}&\textbf{0.39}\\%
				\cline{2%
					-%
					9}%
				&5&\textbf{0.38}&0.34&0.29&0.33&\textbf{0.38}&\textbf{0.38}&\textbf{0.38}\\%
				\cline{2%
					-%
					9}%
				&10&\textbf{0.39}&\textbf{0.39}&0.32&0.35&\textbf{0.39}&\textbf{0.39}&\textbf{0.39}\\%
				\cline{2%
					-%
					9}%
				&20&0.36&0.34&0.34&0.36&\textbf{0.38}&\textbf{0.38}&0.36\\%
				\cline{2%
					-%
					9}%
				&50&\textbf{0.41}&0.32&0.32&0.35&0.4&\textbf{0.41}&\textbf{0.41}\\%
				\hline%
				\multirow{5}{*}{cmc}&{-}&0.36&\textbf{0.39}&0.29&0.26&0.33&\textbf{0.39}&0.36\\%
				\cline{2%
					-%
					9}%
				&5&0.31&0.23&0.2&0.24&\textbf{0.32}&0.31&\textbf{0.32}\\%
				\cline{2%
					-%
					9}%
				&10&0.28&0.15&0.2&0.22&\textbf{0.29}&\textbf{0.29}&0.28\\%
				\cline{2%
					-%
					9}%
				&20&\textbf{0.3}&0.15&0.21&0.26&0.29&\textbf{0.3}&\textbf{0.3}\\%
				\cline{2%
					-%
					9}%
				&50&\textbf{0.31}&0.19&0.23&0.24&0.28&0.3&0.3\\%
				\hline%
				\multirow{5}{*}{hepatitis}&{-}&\textbf{0.56}&0.5&0.53&0.53&\textbf{0.56}&\textbf{0.56}&\textbf{0.56}\\%
				\cline{2%
					-%
					9}%
				&5&0.5&0.47&\textbf{0.53}&0.5&0.5&0.5&0.5\\%
				\cline{2%
					-%
					9}%
				&10&0.47&0.5&\textbf{0.53}&0.47&0.47&0.47&0.47\\%
				\cline{2%
					-%
					9}%
				&20&\textbf{0.59}&0.53&\textbf{0.59}&\textbf{0.59}&\textbf{0.59}&\textbf{0.59}&\textbf{0.59}\\%
				\cline{2%
					-%
					9}%
				&50&\textbf{0.53}&0.5&0.5&\textbf{0.53}&\textbf{0.53}&\textbf{0.53}&\textbf{0.53}\\%
				\hline%
				\multirow{5}{*}{haberman}&{-}&0.28&\textbf{0.32}&0.22&0.2&0.28&0.31&0.28\\%
				\cline{2%
					-%
					9}%
				&5&0.26&0.23&\textbf{0.32}&0.23&0.23&0.26&0.26\\%
				\cline{2%
					-%
					9}%
				&10&0.25&0.23&0.25&\textbf{0.26}&0.2&0.25&0.25\\%
				\cline{2%
					-%
					9}%
				&20&\textbf{0.3}&0.23&0.23&0.26&0.27&\textbf{0.3}&\textbf{0.3}\\%
				\cline{2%
					-%
					9}%
				&50&\textbf{0.31}&0.27&0.3&0.26&0.28&\textbf{0.31}&\textbf{0.31}\\%
				\hline%
				\multirow{5}{*}{glass}&{-}&\textbf{0.24}&0.12&0.12&0.12&\textbf{0.24}&\textbf{0.24}&\textbf{0.24}\\%
				\cline{2%
					-%
					9}%
				&5&0.0&0.0&0.0&0.0&0.0&0.0&0.0\\%
				\cline{2%
					-%
					9}%
				&10&\textbf{0.06}&0.0&\textbf{0.06}&\textbf{0.06}&\textbf{0.06}&\textbf{0.06}&\textbf{0.06}\\%
				\cline{2%
					-%
					9}%
				&20&\textbf{0.06}&\textbf{0.06}&\textbf{0.06}&\textbf{0.06}&\textbf{0.06}&\textbf{0.06}&\textbf{0.06}\\%
				\cline{2%
					-%
					9}%
				&50&0.0&0.0&0.0&0.0&0.0&0.0&0.0\\%
				\hline%
				\multirow{5}{*}{abalone16\_29}&{-}&\textbf{0.33}&0.09&0.11&0.24&0.29&\textbf{0.33}&\textbf{0.33}\\%
				\cline{2%
					-%
					9}%
				&5&\textbf{0.24}&0.08&0.09&0.18&0.21&\textbf{0.24}&\textbf{0.24}\\%
				\cline{2%
					-%
					9}%
				&10&0.18&0.07&0.1&0.15&\textbf{0.23}&0.18&0.18\\%
				\cline{2%
					-%
					9}%
				&20&\textbf{0.18}&0.07&0.09&0.11&\textbf{0.18}&\textbf{0.18}&\textbf{0.18}\\%
				\cline{2%
					-%
					9}%
				&50&0.17&0.08&0.08&0.13&0.16&\textbf{0.18}&0.17\\%
				\hline%
				\multirow{5}{*}{heart\_cleveland}&{-}&\textbf{0.17}&0.03&\textbf{0.17}&\textbf{0.17}&\textbf{0.17}&\textbf{0.17}&\textbf{0.17}\\%
				\cline{2%
					-%
					9}%
				&5&\textbf{0.17}&0.11&0.09&0.11&\textbf{0.17}&\textbf{0.17}&\textbf{0.17}\\%
				\cline{2%
					-%
					9}%
				&10&0.06&0.0&\textbf{0.11}&0.06&0.06&0.06&0.06\\%
				\cline{2%
					-%
					9}%
				&20&0.06&0.0&\textbf{0.11}&0.09&0.06&0.06&0.06\\%
				\cline{2%
					-%
					9}%
				&50&\textbf{0.09}&0.03&\textbf{0.09}&\textbf{0.09}&\textbf{0.09}&\textbf{0.09}&\textbf{0.09}\\%
				\hline%
				\multirow{5}{*}{postoperative}&{-}&\textbf{0.25}&0.08&0.08&0.17&\textbf{0.25}&\textbf{0.25}&\textbf{0.25}\\%
				\cline{2%
					-%
					9}%
				&5&\textbf{0.12}&0.04&0.04&0.08&\textbf{0.12}&\textbf{0.12}&\textbf{0.12}\\%
				\cline{2%
					-%
					9}%
				&10&\textbf{0.12}&0.0&0.08&0.08&\textbf{0.12}&\textbf{0.12}&\textbf{0.12}\\%
				\cline{2%
					-%
					9}%
				&20&\textbf{0.08}&0.04&0.04&\textbf{0.08}&\textbf{0.08}&\textbf{0.08}&\textbf{0.08}\\%
				\cline{2%
					-%
					9}%
				&50&\textbf{0.08}&0.04&0.04&\textbf{0.08}&\textbf{0.08}&\textbf{0.08}&\textbf{0.08}\\%
				\hline%
			\end{tabular}
			}
			\caption{Specyficzność klasy mniejszościowej dla klasyfikatora bagging z drzewem decyzyjnym z ustawieniami $max\_features = 0.9$ oraz $max\_samples = 0.8$.}
			\label{baggingdrzewo2spec2}
		\end{center}
	\end{table}

	
\subsection{Bagging z klasyfikatorem kNN}
Ostatnim klasyfikatorem przetestowanym w metodzie bagging był klasyfikator k najbliższych sąsiadów. W podstawowym ustawieniu, klasyfikator kNN analizuje pięciu sąsiadów. W niniejszym teście postanowiono przetestować meta-klasyfikator bagging z klasyfikatorem kNN dla 1, 2, 3, 5 oraz 7 sąsiadów. Tak samo jak poprzednio, klasyfikatory bagging były budowane z 5, 10, 20 oraz 50 klasyfikatorów. Test ten przeprowadzono z pliku $bagging_knn.py$. W pierwszym etapie budowano modele bagging z wszystkimi atrybutami oraz z maksymalną liczebnością zbiorów. Wyniki dokładności klasyfikacji zostały przedstawione w tabeli \ref{baggingknnacc}. Metoda bagging nie poprawiła jakości klasyfikacji. Dla pojedynczego klasyfikatora kNN czy też grupy klasyfikatorów bagging wyniki są podobne i mieszczą się w granicy błędu. Specyficzność klasy mniejszościowej (tabela \ref{baggingknnspec}) również nie zwiększyła się, pozostała na takim samym poziomie niezależnie od liczby klasyfikatorów składowych. Zmieniając liczbę analizowanych sąsiadów można wpływać na czułość i specyficzność klas. Dla większej liczby sąsiadów np. 5 i 7, rosła skuteczność klasyfikacji oraz czułość klasy większościowej, jednocześnie spadała specyficzność klasy mniejszościowej. Klasyfikator kNN z k większym od 10, nie wykrywał już przykładów klasy zdominowanej. Dla mniejszej liczby sąsiadów (2 i 3) specyficzność była większa. Największą skuteczność w klasyfikacji przykładów z klasy mniejszościowej osiągnięto przy analizowaniu tylko 1 sąsiada. Niestety, spadła wtedy czułość, a precyzja klasy mniejszościowej była niska (duża liczba błędnie zakwalifikowanych przykładów do klasy mniejszościowej).
\begin{table}[H]
	\tiny
		\begin{center}
			\resizebox{\textwidth}{!}{%
				\begin{tabular}{c|c|ccccc}%
					\hline%
					Zbiór danych&Liczba est.&1&2&3&5&7\\%
					\hline%
					\multirow{5}{*}{breast\_cancer}&{-}&0.63&\textbf{0.68}&0.63&0.65&0.65\\%
					\cline{2%
						-%
						7}%
					&5&0.59&0.59&0.57&0.6&\textbf{0.63}\\%
					\cline{2%
						-%
						7}%
					&10&0.63&0.62&0.6&\textbf{0.64}&\textbf{0.64}\\%
					\cline{2%
						-%
						7}%
					&20&0.63&0.62&0.64&0.64&\textbf{0.65}\\%
					\cline{2%
						-%
						7}%
					&50&0.64&0.63&0.63&0.64&\textbf{0.65}\\%
					\hline%
					\multirow{5}{*}{cmc}&{-}&0.71&\textbf{0.76}&0.73&0.74&0.74\\%
					\cline{2%
						-%
						7}%
					&5&0.72&0.74&0.73&\textbf{0.75}&\textbf{0.75}\\%
					\cline{2%
						-%
						7}%
					&10&0.72&0.74&0.74&0.75&\textbf{0.76}\\%
					\cline{2%
						-%
						7}%
					&20&0.71&0.74&0.74&0.75&\textbf{0.76}\\%
					\cline{2%
						-%
						7}%
					&50&0.72&0.73&0.74&0.75&\textbf{0.76}\\%
					\hline%
					\multirow{5}{*}{hepatitis}&{-}&0.65&\textbf{0.77}&0.7&0.7&0.74\\%
					\cline{2%
						-%
						7}%
					&5&0.63&0.71&0.69&0.71&\textbf{0.74}\\%
					\cline{2%
						-%
						7}%
					&10&0.67&0.69&0.71&0.72&\textbf{0.74}\\%
					\cline{2%
						-%
						7}%
					&20&0.68&0.7&0.7&0.72&\textbf{0.74}\\%
					\cline{2%
						-%
						7}%
					&50&0.65&0.7&0.71&0.74&\textbf{0.75}\\%
					\hline%
					\multirow{5}{*}{haberman}&{-}&0.71&\textbf{0.72}&0.68&0.69&0.71\\%
					\cline{2%
						-%
						7}%
					&5&0.69&0.7&\textbf{0.72}&\textbf{0.72}&\textbf{0.72}\\%
					\cline{2%
						-%
						7}%
					&10&0.71&\textbf{0.72}&0.7&0.7&\textbf{0.72}\\%
					\cline{2%
						-%
						7}%
					&20&\textbf{0.71}&\textbf{0.71}&0.7&0.69&\textbf{0.71}\\%
					\cline{2%
						-%
						7}%
					&50&0.71&0.69&0.69&0.69&\textbf{0.72}\\%
					\hline%
					\multirow{5}{*}{glass}&{-}&0.78&0.88&0.84&0.88&\textbf{0.89}\\%
					\cline{2%
						-%
						7}%
					&5&0.79&0.85&0.87&\textbf{0.91}&\textbf{0.91}\\%
					\cline{2%
						-%
						7}%
					&10&0.79&0.82&0.86&\textbf{0.9}&\textbf{0.9}\\%
					\cline{2%
						-%
						7}%
					&20&0.79&0.8&0.84&0.87&\textbf{0.89}\\%
					\cline{2%
						-%
						7}%
					&50&0.78&0.84&0.85&0.87&\textbf{0.89}\\%
					\hline%
					\multirow{5}{*}{abalone16\_29}&{-}&0.92&0.93&0.93&0.93&\textbf{0.94}\\%
					\cline{2%
						-%
						7}%
					&5&0.92&\textbf{0.93}&\textbf{0.93}&\textbf{0.93}&\textbf{0.93}\\%
					\cline{2%
						-%
						7}%
					&10&0.92&\textbf{0.93}&\textbf{0.93}&\textbf{0.93}&\textbf{0.93}\\%
					\cline{2%
						-%
						7}%
					&20&0.92&0.93&0.93&\textbf{0.94}&\textbf{0.94}\\%
					\cline{2%
						-%
						7}%
					&50&0.92&0.93&0.93&\textbf{0.94}&\textbf{0.94}\\%
					\hline%
					\multirow{5}{*}{heart\_cleveland}&{-}&0.83&\textbf{0.88}&0.87&\textbf{0.88}&\textbf{0.88}\\%
					\cline{2%
						-%
						7}%
					&5&0.83&0.86&\textbf{0.87}&\textbf{0.87}&\textbf{0.87}\\%
					\cline{2%
						-%
						7}%
					&10&0.85&0.86&\textbf{0.88}&0.87&\textbf{0.88}\\%
					\cline{2%
						-%
						7}%
					&20&0.84&0.87&\textbf{0.88}&\textbf{0.88}&\textbf{0.88}\\%
					\cline{2%
						-%
						7}%
					&50&0.83&0.86&0.87&\textbf{0.88}&\textbf{0.88}\\%
					\hline%
					\multirow{5}{*}{postoperative}&{-}&0.63&\textbf{0.71}&0.67&0.7&0.69\\%
					\cline{2%
						-%
						7}%
					&5&0.64&0.7&0.67&0.69&\textbf{0.72}\\%
					\cline{2%
						-%
						7}%
					&10&0.68&0.67&0.68&\textbf{0.71}&\textbf{0.71}\\%
					\cline{2%
						-%
						7}%
					&20&0.64&0.67&0.68&0.7&\textbf{0.71}\\%
					\cline{2%
						-%
						7}%
					&50&0.66&0.68&0.67&\textbf{0.71}&0.7\\%
					\hline%
				\end{tabular}}
				\caption{Dokładność klasyfikatora bagging z kNN, dla $max\_features = 1.0$ oraz $max\_samples = 1.0$}
				\label{baggingknnacc}
			\end{center}
		\end{table}
\begin{table}[H]
	\tiny
	\begin{center}
		\resizebox{\textwidth}{!}{%
			\begin{tabular}{c|c|ccccc}%
				\hline%
				Zbiór danych&Liczba est.&1&2&3&5&7\\%
			\hline%
			\multirow{5}{*}{breast\_cancer}&{-}&\textbf{0.36}&0.08&0.28&0.2&0.18\\%
			\cline{2%
				-%
				7}%
			&5&\textbf{0.33}&0.22&0.2&0.15&0.16\\%
			\cline{2%
				-%
				7}%
			&10&\textbf{0.33}&0.27&0.25&0.19&0.16\\%
			\cline{2%
				-%
				7}%
			&20&\textbf{0.33}&0.29&0.26&0.19&0.18\\%
			\cline{2%
				-%
				7}%
			&50&\textbf{0.36}&0.25&0.22&0.14&0.14\\%
			\hline%
			\multirow{5}{*}{cmc}&{-}&\textbf{0.38}&0.17&0.29&0.28&0.24\\%
			\cline{2%
				-%
				7}%
			&5&\textbf{0.35}&0.28&0.3&0.28&0.25\\%
			\cline{2%
				-%
				7}%
			&10&\textbf{0.31}&0.29&0.29&0.27&0.24\\%
			\cline{2%
				-%
				7}%
			&20&\textbf{0.32}&0.3&0.29&0.26&0.24\\%
			\cline{2%
				-%
				7}%
			&50&\textbf{0.32}&0.29&0.28&0.27&0.25\\%
			\hline%
			\multirow{5}{*}{hepatitis}&{-}&\textbf{0.22}&0.03&0.16&0.06&0.0\\%
			\cline{2%
				-%
				7}%
			&5&\textbf{0.19}&0.09&0.12&0.03&0.06\\%
			\cline{2%
				-%
				7}%
			&10&\textbf{0.22}&0.12&0.09&0.09&0.03\\%
			\cline{2%
				-%
				7}%
			&20&\textbf{0.22}&0.12&0.09&0.09&0.03\\%
			\cline{2%
				-%
				7}%
			&50&\textbf{0.22}&0.12&0.09&0.06&0.0\\%
			\hline%
			\multirow{5}{*}{haberman}&{-}&\textbf{0.33}&0.1&0.32&0.25&0.22\\%
			\cline{2%
				-%
				7}%
			&5&0.32&\textbf{0.35}&\textbf{0.35}&0.26&0.27\\%
			\cline{2%
				-%
				7}%
			&10&\textbf{0.31}&0.26&\textbf{0.31}&0.23&0.27\\%
			\cline{2%
				-%
				7}%
			&20&0.31&\textbf{0.32}&0.31&0.2&0.21\\%
			\cline{2%
				-%
				7}%
			&50&\textbf{0.33}&0.3&0.3&0.22&0.23\\%
			\hline%
			\multirow{5}{*}{glass}&{-}&\textbf{0.29}&0.18&0.24&0.18&0.12\\%
			\cline{2%
				-%
				7}%
			&5&\textbf{0.35}&0.24&0.29&0.18&0.0\\%
			\cline{2%
				-%
				7}%
			&10&\textbf{0.29}&0.24&0.24&0.18&0.12\\%
			\cline{2%
				-%
				7}%
			&20&\textbf{0.29}&0.18&0.24&0.18&0.12\\%
			\cline{2%
				-%
				7}%
			&50&\textbf{0.29}&0.18&0.24&0.12&0.12\\%
			\hline%
			\multirow{5}{*}{abalone16\_29}&{-}&\textbf{0.23}&0.08&0.18&0.13&0.1\\%
			\cline{2%
				-%
				7}%
			&5&\textbf{0.23}&0.15&0.16&0.15&0.11\\%
			\cline{2%
				-%
				7}%
			&10&\textbf{0.18}&0.14&0.15&0.1&0.09\\%
			\cline{2%
				-%
				7}%
			&20&\textbf{0.2}&0.17&0.16&0.11&0.09\\%
			\cline{2%
				-%
				7}%
			&50&\textbf{0.22}&0.18&0.16&0.11&0.09\\%
			\hline%
			\multirow{5}{*}{heart\_cleveland}&{-}&\textbf{0.14}&0.0&0.0&0.0&0.0\\%
			\cline{2%
				-%
				7}%
			&5&\textbf{0.11}&0.06&0.09&0.0&0.0\\%
			\cline{2%
				-%
				7}%
			&10&\textbf{0.11}&0.09&0.06&0.0&0.0\\%
			\cline{2%
				-%
				7}%
			&20&\textbf{0.14}&0.11&0.0&0.0&0.0\\%
			\cline{2%
				-%
				7}%
			&50&\textbf{0.14}&0.03&0.0&0.0&0.0\\%
			\hline%
			\multirow{5}{*}{postoperative}&{-}&\textbf{0.21}&0.0&0.08&0.04&0.04\\%
			\cline{2%
				-%
				7}%
			&5&\textbf{0.25}&0.12&0.04&0.04&0.0\\%
			\cline{2%
				-%
				7}%
			&10&\textbf{0.12}&0.04&0.0&0.04&0.0\\%
			\cline{2%
				-%
				7}%
			&20&\textbf{0.12}&0.0&0.0&0.04&0.0\\%
			\cline{2%
				-%
				7}%
			&50&\textbf{0.17}&0.0&0.0&0.04&0.04\\%
			\hline%
			\end{tabular}}
			\caption{Specyficzność klasy mniejszościowej dla klasyfikatora bagging z kNN i ustawieniami $max\_features = 1.0$ oraz $max\_samples = 1.0$.}
			\label{baggingknnspec}
		\end{center}
	\end{table}
\section{Boosting}
Do przetestowania metody boosting wybrano algorytm AdaBoost autorstwa Yoava Freunda i Roberta Schapire w zmodyfikowanej wersji znanej jako AdaBoost-SAMME.R. Algorytm wielokrotnie trenuje "słabe" klasyfikatory na tym samym zbiorze danych, w kolejnych iteracjach zwiększając wagę przykładów źle sklasyfikowanych. Zatem wybrany klasyfikator bazowy musi umożliwiać nadawanie prawdopodobieństwa lub wag przykładom. Jako klasyfikator bazowy, wybrano drzewo decyzyjne oraz naiwny klasyfikator bayesowski. W badaniu zrezygnowano z klasyfikatora kNN, ponieważ nie można w nim nadawać wag przykładom. Budując klasyfikator AdaBoost należy wybrać liczbę iteracji (klasyfikatorów). Testy przeprowadzono dla 5, 10, 15, 30, 50, 100 i 200 iteracji.

\subsection{AdaBoost z naiwnym klasyfikatorem Bayesa}
Test klasyfikatora AdaBoost wykonano w pliku $adaboost\_NB.py$. Badanie przeprowadzono wielokrotnie. Zbudowany klasyfikator był stabilny, otrzymywane wyniki były powtarzalne. Zaobserwowano wzrost dokładności klasyfikacji (tabela \ref{adaboostNBacc2}) w połowie zbiorów danych dla 5, 10 i 15 klasyfikatorów. Dokładności klasyfikacji zwiększyła się średnio o 5\%, dla bazy glass dokładność wzrosła dwa razy a, dla yeastME3 trzy razy. Niestety w pozostałych zbiorach zwiększył się błąd klasyfikacji. Zwiększenie liczby klasyfikatorów (powyżej 15) nie przełożyło się na poprawę wyników, a w większości przypadków dokładność klasyfikacji zmalała. W 16 bazach odnotowany wyraźny spadek specyficzności (tabela \ref{adaboostNKBspec}) (rozpoznawalności) klasy mniejszościowej, a w pozostałych 6 zbiorach wystąpił kilkukrotny wzrost wykrywania klasy zdominowanej. Trzy z sześciu baz zawierały dużą liczbę obserwacji odstających. Potwierdzeniem wszystkich obserwacji są wyniki miary G-mean (tabela \ref{adaboostNKBgmean}). Zazwyczaj wraz ze wzrostem wykrywalności jednej klasy, spada jakość klasyfikacji drugiej klasy. W AdaBoost z NKB zwiększyła się dokładność klasyfikacji klasy dominującej, kosztem klasy zdominowanej. Tylko w 6 zbiorach wystąpił wzrost miary G-mean, a aż w 16 przypadkach osiągnięto gorsze wyniki niż w podstawowej wersji algorytmu NKB.
\begin{table}[H]
	\tiny
	\begin{center}
		\resizebox{\textwidth}{!}{%
		\begin{tabular}{c|cccccccc}%
			Zbiór danych&NB&5&10&15&30&50&100&200\\%
			\hline%
			seeds&0.9&0.69&0.84&0.9&0.89&0.87&\textbf{0.91}&\textbf{0.91}\\%
			new\_thyroid&0.96&0.76&0.94&\textbf{0.97}&0.96&\textbf{0.97}&0.94&0.94\\%
			vehicle&0.66&0.6&0.71&0.64&0.81&0.86&\textbf{0.87}&0.83\\%
			ionosphere&\textbf{0.87}&0.78&0.72&0.69&0.79&0.79&0.83&0.79\\%
			vertebal&\textbf{0.78}&0.73&0.75&0.68&0.67&0.6&0.72&0.72\\%
			yeastME3&0.27&0.47&\textbf{0.87}&0.46&0.75&0.84&0.84&0.84\\%
			ecoli&0.78&0.72&0.89&0.69&0.81&\textbf{0.9}&0.89&0.85\\%
			bupa&0.54&0.53&0.53&0.58&\textbf{0.6}&0.56&0.55&0.58\\%
			horse\_colic&\textbf{0.78}&0.65&0.54&0.55&0.58&0.66&0.69&0.67\\%
			german&\textbf{0.73}&\textbf{0.73}&0.57&\textbf{0.73}&0.57&0.57&0.57&0.57\\%
			breast\_cancer&\textbf{0.72}&0.63&0.36&0.53&0.35&0.35&0.35&0.35\\%
			cmc&0.68&0.66&\textbf{0.73}&0.67&0.63&0.63&0.63&0.63\\%
			hepatitis&0.66&0.63&0.46&\textbf{0.7}&0.61&0.55&0.45&0.6\\%
			haberman&\textbf{0.73}&0.67&0.55&0.48&0.6&0.67&0.69&0.72\\%
			transfusion&\textbf{0.74}&0.56&0.73&0.49&0.41&0.73&0.56&0.58\\%
			car&0.89&\textbf{0.95}&0.9&0.9&0.91&0.91&0.91&0.91\\%
			glass&0.48&\textbf{0.91}&0.78&0.66&0.74&0.79&0.87&0.88\\%
			abalone16\_29&\textbf{0.68}&0.62&0.55&0.62&0.55&0.55&0.55&0.55\\%
			solar\_flare&\textbf{0.65}&\textbf{0.65}&0.32&\textbf{0.65}&0.32&0.32&0.32&0.32\\%
			heart\_cleveland&0.81&0.76&0.82&\textbf{0.88}&0.74&0.8&0.8&0.8\\%
			balance\_scale&\textbf{0.92}&\textbf{0.92}&\textbf{0.92}&\textbf{0.92}&\textbf{0.92}&\textbf{0.92}&\textbf{0.92}&\textbf{0.92}\\%
			postoperative&\textbf{0.67}&0.57&0.63&0.58&0.66&0.54&0.59&0.59\\%
		\end{tabular}}
			\caption{Dokładność klasyfikatora AdaBoost z naiwnym klasyfikatorem Bayesa.}
			\label{adaboostNBacc2}
		\end{center}
	\end{table}

\begin{table}[H]
	\tiny
	\begin{center}
		\resizebox{\textwidth}{!}{%
		\begin{tabular}{c|cccccccc}%
			Zbiór danych&NB&5&10&15&30&50&100&200\\%
			\hline%
			seeds&\textbf{0.91}&0.66&0.64&0.8&0.76&0.81&0.89&0.9\\%
			new\_thyroid&0.87&0.9&0.67&0.83&0.77&\textbf{0.93}&0.6&0.6\\%
			vehicle&\textbf{0.84}&0.78&0.56&0.72&0.56&0.46&0.64&0.78\\%
			ionosphere&\textbf{0.76}&0.7&0.6&0.7&0.64&0.64&0.74&0.49\\%
			vertebal&0.87&0.8&0.61&0.59&0.79&0.79&\textbf{0.9}&0.82\\%
			yeastME3&\textbf{0.99}&0.88&0.29&0.66&0.52&0.49&0.49&0.49\\%
			ecoli&\textbf{0.94}&0.6&0.31&0.51&0.46&0.37&0.49&0.49\\%
			bupa&\textbf{0.74}&0.43&0.6&0.31&0.21&0.27&0.34&0.4\\%
			horse\_colic&\textbf{0.75}&0.43&0.62&0.51&0.36&0.2&0.3&0.33\\%
			german&\textbf{0.62}&\textbf{0.62}&0.32&\textbf{0.62}&0.32&0.32&0.32&0.32\\%
			breast\_cancer&0.44&0.49&\textbf{0.73}&0.66&\textbf{0.73}&0.71&0.71&0.71\\%
			cmc&\textbf{0.61}&0.5&0.03&0.5&0.28&0.28&0.28&0.28\\%
			hepatitis&\textbf{0.78}&0.47&0.44&0.56&0.12&0.38&0.25&0.16\\%
			haberman&0.17&0.23&0.35&\textbf{0.62}&0.3&0.22&0.26&0.23\\%
			transfusion&0.2&0.48&0.3&0.5&\textbf{0.58}&0.31&0.54&0.52\\%
			car&\textbf{1.0}&0.52&\textbf{1.0}&0.42&0.69&0.69&0.69&0.69\\%
			glass&\textbf{0.82}&0.0&0.18&0.18&0.0&0.12&0.06&0.06\\%
			abalone16\_29&0.58&\textbf{0.61}&0.31&\textbf{0.61}&0.31&0.31&0.31&0.31\\%
			solar\_flare&\textbf{0.93}&0.91&0.26&0.91&0.26&0.26&0.26&0.26\\%
			heart\_cleveland&\textbf{0.63}&0.29&0.17&0.03&0.2&0.14&0.14&0.14\\%
			balance\_scale&\textbf{0.0}&\textbf{0.0}&\textbf{0.0}&\textbf{0.0}&\textbf{0.0}&\textbf{0.0}&\textbf{0.0}&\textbf{0.0}\\%
			postoperative&0.17&\textbf{0.46}&0.25&0.33&0.29&0.42&0.42&0.42\\%
		\end{tabular}}
		\caption{Specyficzność klasyfikatora AdaBoost z NKB.}
		\label{adaboostNKBspec}
	\end{center}
\end{table}
\begin{table}[H]
	\tiny
	\begin{center}
		\resizebox{\textwidth}{!}{%
		\begin{tabular}{c|cccccccc}%
			Zbiór danych&NB&5&10&15&30&50&100&200\\%
			\hline%
			seeds&\textbf{0.91}&0.68&0.78&0.88&0.85&0.86&\textbf{0.91}&\textbf{0.91}\\%
			new\_thyroid&0.92&0.82&0.81&0.91&0.87&\textbf{0.96}&0.77&0.77\\%
			vehicle&0.72&0.65&0.65&0.67&0.7&0.67&0.78&\textbf{0.81}\\%
			ionosphere&\textbf{0.84}&0.76&0.68&0.69&0.75&0.75&0.81&0.68\\%
			vertebal&\textbf{0.8}&0.74&0.7&0.65&0.69&0.63&0.75&0.74\\%
			yeastME3&0.42&0.61&0.52&0.53&0.64&\textbf{0.66}&\textbf{0.66}&\textbf{0.66}\\%
			ecoli&\textbf{0.85}&0.66&0.55&0.6&0.62&0.6&0.67&0.66\\%
			bupa&\textbf{0.55}&0.51&0.53&0.49&0.42&0.45&0.49&0.54\\%
			horse\_colic&\textbf{0.77}&0.58&0.55&0.54&0.51&0.43&0.53&0.54\\%
			german&\textbf{0.69}&\textbf{0.69}&0.46&\textbf{0.69}&0.46&0.46&0.46&0.46\\%
			breast\_cancer&\textbf{0.6}&0.58&0.39&0.56&0.37&0.37&0.37&0.37\\%
			cmc&\textbf{0.65}&0.59&0.18&0.6&0.45&0.45&0.45&0.45\\%
			hepatitis&\textbf{0.7}&0.56&0.45&0.65&0.3&0.47&0.35&0.33\\%
			haberman&0.4&0.44&0.46&\textbf{0.52}&0.46&0.43&0.47&0.46\\%
			transfusion&0.43&0.53&0.51&0.49&0.45&0.52&0.55&\textbf{0.56}\\%
			car&0.94&0.71&\textbf{0.95}&0.62&0.8&0.8&0.8&0.8\\%
			glass&\textbf{0.61}&0.0&0.38&0.35&0.0&0.32&0.24&0.24\\%
			abalone16\_29&\textbf{0.63}&0.61&0.42&0.61&0.42&0.42&0.42&0.42\\%
			solar\_flare&\textbf{0.77}&0.76&0.29&0.76&0.29&0.29&0.29&0.29\\%
			heart\_cleveland&\textbf{0.72}&0.48&0.39&0.17&0.4&0.36&0.36&0.36\\%
			balance\_scale&\textbf{0.0}&\textbf{0.0}&\textbf{0.0}&\textbf{0.0}&\textbf{0.0}&\textbf{0.0}&\textbf{0.0}&\textbf{0.0}\\%
			postoperative&0.38&\textbf{0.53}&0.44&0.47&0.48&0.5&0.52&0.52\\%
		\end{tabular}}
			\caption{Miara G-mean dla AdaBoost z NKB}
			\label{adaboostNKBgmean}
		\end{center}
	\end{table}
	
\subsection{AdaBoost z drzewem decyzyjnym}

\subsection{Stacking}

\todo{wielokrotne testy}
\todo{porównać wyniki ada, stacking, boosting}
\setlength{\belowcaptionskip}{-5pt}
\chapter{Propozycja klasyfikatorów}
\section{Klasyfikator ekspercki}
Klasyfikator ekspercki powstał na bazie doświadczeń z podstawowymi klasyfikatorami. W zależności o charakterystyki danych, osiągana skuteczność przez klasyfikatory może się różnić. Klasyfikator skutecznie rozpoznający klasy jednego zbioru, może miernie klasyfikować inny zbiór, podczas gdy użycie innego klasyfikatora na tym samym zbiorze danych może znacząco poprawić osiągane wyniki. Również użycie różnych algorytmów klasyfikacji, połączenie ich w komitet może zmniejszyć błąd klasyfikacji. 
Klasyfikator ekspercki został stworzony w celu zmniejszenia błędu klasyfikacji niezależnie od typu danych oraz nieznanych danych. \par
\begin{figure}[h]
	\centering
	\includegraphics[width=\textwidth]{./images/klas_ekspercki.png}
	\caption{Schemat klasyfikatora eksperckiego. $KB_1..KB_i$ to klasyfikatory bazowe.}
	\label{fig:klasyfikator_ekspercki}
\end{figure}
Klasyfikator ekspercki to połączenie kilku klasyfikatorów bazowych (przynajmniej trzech). Każdy klasyfikator trenowany jest na zbiorze uczącym, a następnie oceniana jest jakość klasyfikacji każdego z osobna. Stworzono dwie wersje klasyfikatora, w pierwszej klasyfikator uczony i testowany jest na tym samym zbiorze danych, w drugiej oceniany jest z wykorzystaniem sprawdzianu krzyżowego (domyślnie k=3). W kolejnym etapie, dla każdej klasy wyłaniany jest klasyfikator ekspert. Ekspert klasowy wybierany jest na podstawie najwyższego współczynnika dla danej klasy. Tworząc klasyfikator można wybrać na podstawie którego współczynnika precyzji, F1 czy G-mean będą wyłaniani eksperci. Domyślnie jest to współczynnik precyzji. Przy wyborze miary G-mean ekspert dla obu będzie taki sam. Jeżeli ocena odbywała się ze sprawdzianem krzyżowym, to modele bazowe tworzone są od nowa na całym zbiorze treningowym. \par
W procesie klasyfikacji właściwej nowych próbek, najpierw klasyfikowane są przez klasyfikatory bazowe. Końcowa klasa wyznaczana jest według algorytmu:
\begin{enumerate}
	\item Jeżeli występuje zgodność co do klasy pomiędzy klasyfikatorami to wybierana jest ta klasa.
	\item Jeżeli tylko jeden ekspert wskaże swoją klasę to ostateczną klasą jest ta wskazana przez eksperta.
	\item Jeżeli dwóch ekspertów wskażą swoje klasy, to wybierana jest klasa z większym prawdopodobieństwem wskazanym przez klasyfikator. W przypadku takich samych prawdopodobieństw, wybierana jest klasa wskazana przez klasyfikator z większym współczynnikiem G-mean.
	\item Jeżeli żaden ekspert nie wskaże swojej klasy, to klasa wybierana jest poprzez głosowanie większościowe
\end{enumerate}
\subsection{Testy}

\section{Meta-klasyfikator}

\begin{figure}[H]
	\centering
	\includegraphics[width=\textwidth]{./images/metaklas.png}
	\caption{Projekt meta klasyfikatora}
	\label{fig:metaklasmoj}
\end{figure}
\subsection{Testy}
\chapter{Podsumowanie}
W pracy przedstawiono algorytmy klasyfikacji oraz trzy meta-metody: bagging, boosting, stacking. Pokazano w~jaki sposób należy oceniać klasyfikator. Przeprowadzono badania jakości klasyfikacji z~użyciem meta-metod dla kilku różnych algorytmów klasyfikacji. Do testów użyto zróżnicowane (pod względem liczby przykładów, atrybutów, liczby atrybutów kategorycznych, różnym stopniu niezrównoważenia klas) i~prawdziwe zbiory danych. Analiza otrzymanych wyników oraz porównanie ich do podstawowych klasyfikatorów wykazała, że z~wykorzystaniem meta-metod można podnieść jakość klasyfikacji. Najlepszymi meta-klasyfikatorami okazały~się bagging z~naiwnym klasyfikatorem Bayesa oraz drzewem decyzyjnym. Algorytm AdaBoost z~drzewem decyzyjnym uzyskał minimalnie gorsze wyniki. Stacking natomiast był najbardziej stabilnym meta-klasyfikatorem, uzyskując wysoką skuteczność klasyfikacji dla większości baz danych. Użycie klasyfikatora kNN, jako klasyfikatora składowego meta-metod, nie poprawiło jakości klasyfikacji.\par
Meta-metody z~różnym skutkiem klasyfikowały klasę mniejszościową. Użycie metod równoważenia liczebności klas w~zbiorach z~meta-metodami pozwoliło zwiększyć wykrywalność klasy mniejszościowej o~kilka procent, a~w niektórych przypadkach kilkukrotnie. Najbardziej skutecznymi metodami okazały~się metoda ADASYN oraz metoda NCR. \par
W ramach pracy zaprojektowano klasyfikator ekspercki oraz meta-klasyfikator. Przeprowadzono testy klasyfikatora eksperckiego, które wykazały lepszą lub taką samą dokładność klasyfikacji jak drzewo decyzyjne w~większości zbiorów danych. W prawie wszystkich zbiorach poprawił lub utrzymał skuteczność klasyfikacji klasy mniejszościowej. Wyniki stworzonego meta-klasyfikatora były porównywalne z~wynikami meta-metod. Kluczową cechą zbudowanego meta-klasyfikatora miała być uniwersalność. Niezależnie od danych miał uzyskiwać wysokie wyniki. Analiza wyników wykazała, że meta-klasyfikator dobrze łączy przewidywania klasyfikatorów składowych i~uzyskuje wysoką skuteczność klasyfikacji. Uzyskane wyniki plasowały~się w~czołówce dla wszystkich zbiorów danych. Meta-klasyfikator uzyskał większą jakość klasyfikacji klasy mniejszościowej od większości meta-metod. \par
Mimo złożoności meta-metod, meta-klasyfikatora i~konieczności budowania wielokrotnie różnych modeli klasyfikacyjnych to czas potrzebny na stworzenie modelu klasyfikacyjnego jest niski (dla przetestowanych zbiorów danych). Klasyfikacja nowych przykładów odbywała~się bardzo szybko. \par
Badania nad skutecznością meta-metod można kontynuować i~przeprowadzić z~użyciem innych algorytmów klasyfikacji. Zastosowanie bardziej różnorodnych algorytmów klasyfikacji lub większej ilości klasyfikatorów w~metodzie stacking powinno pozwolić na zwiększenie jakości klasyfikacji. Interesującym zagadnieniem może być klasyfikacja niezrównoważonych zbiorów danych z~wykorzystaniem zmodyfikowanych meta-metod, zawierającymi w~sobie algorytmy równoważenia liczebności zbiorów. Kolejnym pomysłem na poprawienie skuteczności klasyfikacji, może być dobór algorytmu w~zależności od charakterystyki danych.


\begin{thebibliography}{99}
\addcontentsline{toc}{chapter}{Bibliografia}
\bibitem{KubatMatwin}{M. Kubat i S. Matwin. Addresing the curse of imbalanced training sets: one-side selection.}
\bibitem{Garcia}{H.	He i E.	A. Garcia. Learning from imbalanced data. IEEE Transactions on Data and Knowledge Engineering, 2009.}
\bibitem{gocardless}{N. Hockham: Machine learning with imbalanced data sets https://www.youtube.com/watch?v=X9MZtvvQDR4}
\bibitem{Bishop}{C. M. Bishop. Neural Networks for Pattern Recognition. Claredon press. Oxford, 1995.}
\bibitem{boosting}{Long P, Servedio R. Random Classification Noise Defeats All Convex Potential Boosters}
\bibitem{uci}{The UCI Machine Learning Repository, https://archive.ics.uci.edu/ml/}
\bibitem{hyper}{F. Hu, X. Liu, J. Dai i H. Yu. A Novel Algorithm for Imbalance Data Classification Based on Neighborhood Hypergraph}
\bibitem{StefImbalanced}{J. Stefanowski. Dealing with Data Difficulty Factors while Learning from Imbalanced Data.}
\end{thebibliography}

\end{document}