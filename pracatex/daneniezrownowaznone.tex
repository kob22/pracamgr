\chapter{Klasyfikacja danych niezrównoważonych}
\todo{wspomnieć o 2 podejścia -> preprecesing oraz algorytmy (algorytmamie się nie zajmuje)}

Większość istniejących algorytmów klasyfikacji, nastawiona jest na poprawną klasyfikację zbiorów o zrównoważonej liczebności wszystkich klas. Niestety, w rzeczywistych problemach, bardzo często zdarza się, że zbiory są mocno niezbilansowane.
\section{Dane niezrównoważone}
\todo{dodac odnosnik do one vs all}
Dane są niezrównoważone (ang. \textit{imbalanced data} jeśli klasy decyzyjne nie są przybliżeniu tak samo liczebne. Najmniejsza klasa, nazywana jest klasą mniejszościową (ang. \textit{minority class}), natomiast klasa dominująca, lub pozostałe połączone klasy (można połączyć pozostałe klasy w jedną, doprowadzając do klasyfikacji binarnej, one vs all), nazywana jest klasą większościową (ang. \textit{majority class}). W praktyce klasa mniejszościowa, zazwyczaj liczy około 10-20\% wszystkich przykładów. Często zdarzają się jednak takie problemy, gdzie to zróżnicowanie jest większe np.:
\begin{itemize}
	\item około 2\% transakcji kartami kredytowymi w GOCARDLESS to oszustwa \cite{gocardless}.
	
\end{itemize}
\todo{dodac przyklady danych mniejszosciowych}
W przytoczonych przykładach ważniejsza jest klasa mniejszościowa i wykrycie jej stanowi priorytet. Niezrównoważenie klas w zbiorze danych stanowi problem w fazie uczenia i znacząco obniża jakość klasyfikacji. Ze względu na częstość występowania klasy dominującej, klasyfikator preferuje tą klasę, dążąc do optymalizacji i obniżenia błędu error rate (\ref{error_rate}) nie biorąc pod uwagę rozłożenia klas w zbiorze. Klasyfikator może osiągnąć wysoką skuteczność klasyfikacji np. 95\% przy niskiej lub zerowej wykrywalności klasy mniejszościowej. 
Należy oczekiwać od klasyfikatora wysokiej skuteczności wykrywania klasy mniejszościowej, nawet kosztem pogorszenia rozpoznawania klasy większościowej.
Poddając analizie sąsiedztwa przykłady z klasy zdominowanej, można wyróżnić przykłady bezpieczne i niebezpieczne w klasyfikacji:
\begin{itemize}
	\item safe - przykład bezpieczny, w jego sąsiedztwie zdecydowana większość obserwacji jest z tej samej klasy,
	\item borderline - graniczny, przykład niebezpieczny, w jego sąsiedztwie ilość przykładów z obu klas jest podobna
	\item outlier - poboczny, przykład niebezpieczny, w jego sąsiedztwie większość obserwacji jest z klasy przeciwnej, dominującej,
	\item rare - rzadki, przykład niebezpieczny, w jego sąsiedztwie występują tylko przykłady z klasy przeciwnej, większościowej.
\end{itemize}
\todo{dodac obrazek danyych safe border itd.}
\todo{podac wykresy przykłady danych niezrównoważonych}



\section{!!preprocessing danych niezrównoważonych}
W celu zrównoważenia rozkładu danych niezbilansowanych wprowadzono różne metody usuwania przykładów klasy dominującej lub tworzenia sztucznych obserwacji klasy mniejszościowej. Poniżej zostaną omówione metody, które zostały użyte podczas badań.
\subsection{Metody undersampling}
Jest to cała rodzina różnych metod, które usuwają przykłady z klasy większościowej. \textbf{Losowe usuwanie} (ang. \textit{random undersampling}), jak sama nazwa wskazuje, usuwa losowo przykłady z klasy dominującej. Rozwiązanie to ma niestety wadę. Jeśli usunie się zbyt dużo przykładów danego przypadku, można pozbawić klasyfikator bardzo ważnej informacji. \par
Lepszym rozwiązaniem jest świadome usuwanie przykładów spełniających określone kryteria. Taką metodą jest \textbf{undersampling z "Tomek links"}. Parę punktów Tomek link, definiuje się jako dwa punkty należące do różnych klas, z odległością równą $d(E_i,E_j)$, jeśli nie istnieje inny punkt $E_l$, taki, że $d(E_i,E_l) < d(E_i,E_j)$ lub $d(E_j,E_l) < d(E_i,E_j)$. Punkty tworzące Tomek link to szum lub punkt graniczny. Po znalezieniu takich punktów, usuwa się przykład z klasy dominującej. Usunięcie takiej obserwacji, powoduje rozszerzenie granicy klasy mniejszościowej. \par
\todo{opisac reszte te ktore wykorzystam}

\subsection{Metody oversampling}