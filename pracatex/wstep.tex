\chapter*{Wstęp}
\addcontentsline{toc}{chapter}{Wstęp}
Ostatnie dekady to wielki skok cywilizacyjny. Nastąpił gwałtowny rozwój informatyki oraz pokrewnych dziedzin. Wskutek postępu technicznego, ludzie zaczęli generować ogromne ilości danych, które zostają zapisane na co raz większych i~tańszych nośnikach danych. Codziennie, podczas zwykłych czynności np. płacenia kartą w~sklepie, przeglądania internetu, wysyłania poczty e-mail, nosząc ze sobą podłączony do sieci GSM telefon tworzymy nieświadomie duże ilości danych. Firmy ubezpieczeniowe, banki, sklepy, sieci handlowe i~inne instytucje prywatne oraz publiczne zbierają i~przetwarzają o nas różne informacje, a~następnie gromadzą je w~ogromnych bazach danych. Dziedzinami zajmującymi się przetwarzaniem danych oraz wyszukiwaniem wzorców i~wiedzy w~danych jest uczenie maszynowe (ang. \textit{machine learning}) oraz eksploracja danych (ang. \textit{data mining}). Uczenie maszynowe można podzielić między innymi na uczenie nadzorowane (ang. \textit{supervised learning}) oraz uczenie nienadzorowane (ang. \textit{unsupervised learning}). W uczeniu nadzorowanym program dla znanych danych z przypisanymi klasami lub kategoriami tworzy reguły decyzyjne, które następnie wykorzystywane są do przypisania klasy nowym nieznanym wcześniej danym. Natomiast w~uczeniu nienadzorowanym, wyszukiwane są wzorce, struktury w~danych bez przypisanej końcowej kategorii. \par 
Elementem uczenia nadzorowanego jest klasyfikacja danych (ang. \textit{classification}). W procesie klasyfikacji danych, system wyszukuje wzorców oraz reguł w~danych uczących ze znanymi kategoriami, a~następnie przypisuje kategorię nowym obserwacjom bez określonej klasy. Oczywiście przypisana kategoria pochodzi ze zbioru danych uczących. Przykładem klasyfikacji danych może być proces przyznania kredytu. Bank zbiera różne informacje o klientach (rodzaj pracy, umowy, wysokość zarobków, poprzednie pożyczki, kredyty, terminowość spłat rat), a~następnie po jakimś czasie oznacza czy był to dobry klient z korzyścią dla banku. Klasyfikator nauczony tymi danymi, określa czy można udzielić kredytu nowemu klientowi na podstawie dostarczonych danych. Innym przykładem mogą być dane medyczne np. osób chorych na raka. Lekarz gromadzi parametry medyczne pacjentów wraz z kategorią czy dana osoba jest chora. Na podstawie tych danych, tworzony jest model klasyfikacyjny, który pomaga określić czy nowy pacjent może być potencjalnie chory.\par 
Skuteczność klasyfikacji zależna jest między innymi od ilości i~jakości danych oraz od klasyfikatora (algorytmu klasyfikującego). Istnieje wiele różnych algorytmów klasyfikacji, takich jak: drzewo decyzyjne, naiwny klasyfikator Bayesa, sieć neuronowa, maszyna wektorów nośnych oraz wiele innych. Algorytmy te osiągają różną skuteczność dla różnych danych. W celu poprawy skuteczności klasyfikacji można wykorzystać meta-metody, inaczej metauczenie (ang. \textit{meta learning}). Celem metauczenia jest poprawa skuteczności klasyfikacji istniejących algorytmów. Najczęściej wykorzystuje się przewidywania różnych klasyfikatorów, które łączy się tworząc metadane. Następnie meta-klasyfikator (główny klasyfikator) lub wszystkie klasyfikatory poprzez głosowanie decydują o końcowej klasie nowych obserwacji. 	\par 
Dane medyczne często zawierają dużą ilość osób zdrowych, a~małą ilość osób chorych. Klasyfikując nowych pacjentów zależy nam na dużej skuteczności wykrywania osób potencjalnie chorych. Niestety, w~przypadku dużego niezrównoważenia klas danych, klasyfikator często osiąga niską skuteczność klasyfikacji mniej liczebnej klasy. Aby poprawić wykrywalność mniejszej klasy stosuje się sztuczne równoważenie zbiorów lub zmodyfikowane algorytmy klasyfikacji.