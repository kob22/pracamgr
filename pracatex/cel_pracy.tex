\chapter*{Zakres i~cel pracy}
\addcontentsline{toc}{chapter}{Zakres i~cel pracy}
Celem pracy magisterskiej było zbadanie wpływu meta-metod (metauczenia) na poprawę jakości klasyfikacji danych oraz danych niezrównoważonych. Do badań wybrano algorytmy bagging, boosting oraz stacking. Wszystkie meta-metody miały zostać przetestowane z kilkoma podstawowymi algorytmami klasyfikacji. Klasyfikacja miała być przeprowadzona dla danych z różnych dziedzin, z rożną liczbą przykładów i~atrybutów oraz z różnym rozkładem klas. Ważnym elementem badań była klasyfikacja zbiorów niezrównoważonych, z wyraźną dominacją liczebną jednej klasy. We wstępnym przetwarzaniu danych miały zostać użyte metody równoważące rozkład klas w~zbiorze danych. Metody równoważenia klas danych, należało porównać i~zbadać wpływ na proces klasyfikacji z meta-metodami. Ostatnim elementem niniejszej pracy była próba stworzenia uniwersalnego meta-klasyfikatora, uzyskującego wysoką jakość klasyfikacji dla większości zbiorów danych. \par
Zakres pracy stanowiło przetestowanie meta-metod. Gotowe algorytmy klasyfikacji miały pochodzić z biblioteki $scikit-learn$. Implementacja algorytmów klasyfikacji nie wchodziła w~zakres pracy. Miały zostać napisane testy w~języku $python$ badające, oceniające i~porównujące meta-metody oraz wybrane klasyfikatory. Testy powinny zwracać otrzymane wyniki w~postaci plików pdf. Ostatnim elementem było stworzenie meta-klasyfikatora. \par
Rozdział pierwszy i~drugi stanowią wprowadzenie teoretyczne do problemu klasyfikacji oraz klasyfikacji danych niezrównoważonych. W rozdziałach tych opisana jest klasyfikacja, użyte algorytmy klasyfikacji oraz meta-metody. Czytelnik może zaznajomić się także z informacją w~jaki sposób i~z jakimi miarami ocenia się klasyfikację danych. W rozdziale poświęconym klasyfikacji danych niezrównoważonych, znajduje się geneza problemu oraz opis dostępnych i~możliwych rozwiązań poprawy klasyfikacji takich danych. \par
Rozdział trzeci poświęcony jest opisowi platformy do badań oraz analizie użytych danych do klasyfikacji. Zostały także przetestowane i~wybrane wiarygodne sposoby oceniania klasyfikatora oraz sposoby równoważenia zbiorów danych. \par
W rozdziale czwartym opisano przeprowadzone badania na meta-metodach. Przeprowadzono testy z różnymi algorytmami oraz ustawieniami klasyfikatorów. Część rozdziału przeznaczona została na poprawę klasyfikacji danych mniejszościowych. Przetestowano i~porównano wybrane metody równoważenia klas w~zbiorach razem z meta-metodami. \par
W piątym rozdziale zaprezentowano zmodyfikowany pomysł głosowania, w~postaci klasyfikatora eksperckiego. Przedstawiono także projekt meta-klasyfikatora. Podobnie jak poprzednio, oba klasyfikatory zostały przetestowane z użyciem prawdziwych zbiorów danych. \par
W ostatnim rozdziale opisano uzyskane wyniki oraz podsumowanie całej pracy.