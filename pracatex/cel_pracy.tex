\chapter*{Zakres i cel pracy}
\addcontentsline{toc}{chapter}{Zakres i cel pracy}
Celem pracy magisterskiej było zbadanie wpływu meta-metod (metauczenia) na poprawę jakości klasyfikacji danych oraz danych niezrównoważonych. Do badań wybrano algorytmy bagging, boosting oraz stacking. Wszystkie meta-metody miały zostać przetestowane z kilkoma podstawowymi algorytmami klasyfikacji. Klasyfikacja miała być przeprowadzona dla danych z różnych dziedzin, z rożną ilością przykładów i atrybutów oraz z różnym rozkładem klas. Ważnym elementem badań było zbadanie wpływu wstępnego przetwarzania danych (równoważenia zbiorów) na skuteczność klasyfikacji. 