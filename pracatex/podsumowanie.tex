\chapter{Podsumowanie}
W pracy przedstawiono algorytmy klasyfikacji oraz trzy meta-metody: bagging, boosting, stacking. Pokazano w~jaki sposób należy oceniać klasyfikator. Przeprowadzono badania jakości klasyfikacji z~użyciem meta-metod dla kilku różnych algorytmów klasyfikacji. Do testów użyto zróżnicowane (pod względem liczby przykładów, atrybutów, liczby atrybutów kategorycznych, różnym stopniu niezrównoważenia klas) i~prawdziwe zbiory danych. Analiza otrzymanych wyników oraz porównanie ich do podstawowych klasyfikatorów wykazała, że z~wykorzystaniem meta-metod można podnieść jakość klasyfikacji. Najlepszymi meta-klasyfikatorami okazały~się bagging z~naiwnym klasyfikatorem Bayesa oraz drzewem decyzyjnym. Algorytm AdaBoost z~drzewem decyzyjnym uzyskał minimalnie gorsze wyniki. Stacking natomiast był najbardziej stabilnym meta-klasyfikatorem, uzyskując wysoką skuteczność klasyfikacji dla większości baz danych. Użycie klasyfikatora kNN, jako klasyfikatora składowego meta-metod, nie poprawiło jakości klasyfikacji.\par
Meta-metody z~różnym skutkiem klasyfikowały klasę mniejszościową. Użycie metod równoważenia liczebności klas w~zbiorach z~meta-metodami pozwoliło zwiększyć wykrywalność klasy mniejszościowej o~kilka procent, a~w niektórych przypadkach kilkukrotnie. Najbardziej skutecznymi metodami okazały~się metoda ADASYN oraz metoda NCR. \par
W ramach pracy zaprojektowano klasyfikator ekspercki oraz meta-klasyfikator. Przeprowadzono testy klasyfikatora eksperckiego, które wykazały lepszą lub taką samą dokładność klasyfikacji jak drzewo decyzyjne w~większości zbiorów danych. W prawie wszystkich zbiorach poprawił lub utrzymał skuteczność klasyfikacji klasy mniejszościowej. Wyniki stworzonego meta-klasyfikatora były porównywalne z~wynikami meta-metod. Kluczową cechą zbudowanego meta-klasyfikatora miała być uniwersalność. Niezależnie od danych miał uzyskiwać wysokie wyniki. Analiza wyników wykazała, że meta-klasyfikator dobrze łączy przewidywania klasyfikatorów składowych i~uzyskuje wysoką skuteczność klasyfikacji. Uzyskane wyniki plasowały~się w~czołówce dla wszystkich zbiorów danych. Meta-klasyfikator uzyskał większą jakość klasyfikacji klasy mniejszościowej od większości meta-metod. \par
Mimo złożoności meta-metod, meta-klasyfikatora i~konieczności budowania wielokrotnie różnych modeli klasyfikacyjnych to czas potrzebny na stworzenie modelu klasyfikacyjnego jest niski (dla przetestowanych zbiorów danych). Klasyfikacja nowych przykładów odbywała~się bardzo szybko. \par
Badania nad skutecznością meta-metod można kontynuować i~przeprowadzić z~użyciem innych algorytmów klasyfikacji. Zastosowanie bardziej różnorodnych algorytmów klasyfikacji lub większej ilości klasyfikatorów w~metodzie stacking powinno pozwolić na zwiększenie jakości klasyfikacji. Interesującym zagadnieniem może być klasyfikacja niezrównoważonych zbiorów danych z~wykorzystaniem zmodyfikowanych meta-metod, zawierającymi w~sobie algorytmy równoważenia liczebności zbiorów. Kolejnym pomysłem na poprawienie skuteczności klasyfikacji, może być dobór algorytmu w~zależności od charakterystyki danych.
